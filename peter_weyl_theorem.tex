\documentclass{report}

\usepackage{style}

\begin{document}
\tableofcontents
\chapter{Einführung}
\section{Motivation}
Das zentrale Ergebnis des letzten Vortrags war die Existenz
und Eindeutigkeit des Haarmaßes, welche sich im folgenden
Satz zusammenfassen lässt.
\begin{satz*}[Existenz und Eindeutigkeit des Haarmaßes]
    Sei $G$ eine lokalkompakte Gruppe. Dann gibt es eine 
    linksinvariantes Radonmaß $ \mu\neq 0$ auf 
    $\mathcal{B}(X)$. Dieses ist eindeutig bis auf 
    positive Vielfache.
\end{satz*}
Dieses Maß wird als das Haarmaß bezeichnet. Somit erhalten
wir auf jeder Gruppe einen sinnvollen Begriff des Integrals.
Dies erlaubt uns, wirklich sinnvoll über Räume wie 
$\LL^2(G)$ zu reden.\\
Im Spezialfall erhalten wir dann natürlich Ergebnisse für
die Integrationstheorie auf dem $\R^d$ oder anderen 
spannenden Gruppen wie $\R^\times$. Jedoch sind nicht alle
Gruppen so gut strukturiert wie diese Gruppen. Diese sind
nämlich abelsch und lokalkompakt. Die Integrationstheorie
auf solchen Gruppen ist zentraler Gegenstand der 
\emph{harmonischen Analysis}. Der nicht abelsche Fall ist
jedoch nicht so gut strukturiert und deutlich 
unübersichtlicher. Für den kompakten Fall erhalten wir 
jedoch den \emph{Satz von Peter-Weyl}, welcher viele 
Informationen über die Struktur von $\LL^2(G)$ beinhaltet.
Diesen zu verstehen und zu beweisen ist Gegenstand des 
vorliegenden Seminarvortrags.\\
Zunächst brauchen wir jedoch noch Grundlagen der
Darstellungstheorie topologischer Gruppen.
\section{Notation}
\begin{itemize}
    \item Für einen Banachraum $V$ bezeichnen wir mit 
        $\GLS{V}$ die Menge der stetigen, linearen, 
        invertierbaren Abbildungen.
    \item Für zwei Banachräume $V, W$ bezeichnen wir mit
        $ \mathcal L(V, W)$ die Menge der beschränkten,
        stetigen linearen Abbildungen $T:V\to W$.
\end{itemize}
\chapter{Grundlagen}
\section{Haarsche Integrationstheorie}
\section{Funktionalanylsis}
\chapter{Darstellungstheorie}
Ziel der Darstellungstheorie ist es gewissermaßen, das 
Problem der schwierigen Strukturierung von Gruppen zu
lösen, indem man die Elemente in lineare Operatoren 
übersetzt.
Für diese kennen wir nämlich aus der linearen Algebra im 
endlichdimensionalen sowie der Funktionalanalysis um 
unendlichdimensionalen Fall viele hilfreiche Sätze, mit 
denen wir sehr häufig mehr über die Struktur der Gruppe 
verstehen können.
\section{Grundlegende Begriffe}
Der zentrale Begriff in der Darstellungstheorie 
topologischer Gruppen ist der einer Darstellung.
\begin{definition}[Darstellung]
Sei $V$ ein Banachraum und $G$ eine topologische Gruppe. 
Eine \underline{Banachraum-Darstellung}
\index{Banachraum-Darstellung}\footnote{Häufig werden wir 
nur Darstellung sagen und damit eine Banachraum-Darstellung 
meinen.} von $G$ auf $V$ ist ein Gruppenhomomorphismus 
$ \pi: G\to \GLS{V}$, sodass die Abbildung $$G\times V\to V,
(g,v)\mapsto (g,v)\mapsto \pi(g)v$$ stetig ist.
\end{definition}
\begin{beispiel}
Für $G=\SL(2, \R)$ erhalten wir folgende 
Standarddarstellung auf $V= \R^2$ gegeben durch $\pi: 
\SL(2,\R)\to \GL(\R^2), A\mapsto A$ mit 
Matrixmultiplikation als zugehöriger Wirkung.
\end{beispiel}
Bereits aus der Algebra 1 wissen wir um die Wichtigkeit 
bestimmter "elementarer" Strukturen, z.B. den zyklischen
Gruppen bei der Klassifikation endlicher abelscher Gruppen.
Auch in der Darstellungstheorie gibt es einen ähnlichen
Begriff:
\begin{definition}[Unterdarstellungen, irreduzible 
    Darstellungen]
Sei $( \rho, V_ \rho)$ eine Darstellung einer topologischen
    Gruppe $G$. \begin{enumerate}
    \item $( \pi, V_ \pi)$ heißt \underline{
        Unterdarstellung}\index{Unterdarstellung} von 
        $ ( \rho, V_ \rho)$, falls $ V_ \pi$ ein 
        abgeschlossener Untervektorraum von $ V_ \rho$
        ist und $ \rho$ eingeschränkt auf $ V_ \pi$
        gleich $ \pi$ ist.
    \item $( \rho, V_ \rho)$ heißt \underline{irreduzibel}
        \index{irreduzibel}, falls es keine nicht-trivialen
        Unterdarstellungen gibt, also falls für jede 
        Unterdarstellung $( \pi, V_ \pi)$ gilt, dass
        $V_ \pi = \{0\}$ oder $ V_ \pi = V_ \rho$.
\end{enumerate}
\end{definition}
\begin{bemerkung}
Jeder abgeschlossene Untervektorraum $U \subseteq V_ \rho$, 
für den $ \rho(G) U \subseteq U$ gilt, liefert eine 
Unterdarstellung durch Einschränken von $ \rho$ auf $ U$.
\end{bemerkung}

Für den Fall, dass die Banachräume $V$ für unsere 
Darstellungen sogar Hilberträume sind, gibt es noch
eine weitere Art von Darstellungen:
\begin{definition}[Unitäre Darstellungen]
Sei $V$ ein Hilbertraum und $G$ eine topologische Gruppe.
Eine Darstellung $ \rho$ auf $ V$ heißt \underline{unitäre 
Darstellung}\index{unitäre Darstellung}, falls 
$ \rho(g)$ unitär ist für alle $g \in G$, also
$ \langle \rho(g)v, \rho(g)w \rangle = \langle v, w\rangle$
für alle $ g \in G$ und alle $ v,w \in V$.
\end{definition}
Wir möchten später zeigen, dass sich jede endlichdimensionale
Darstellung als direkte Summe von irreduziblen Darstellungen
schreiben lässt. Dazu definieren wir
\begin{definition}[Direkte Summe von Darstellungen]
Seien $( \rho_1, V_1), (\rho_2, V_2)$ zwei unitäre 
Darstellungen. Auf der direkten Summe von Vektorräumen
$V \coloneq V_1 \oplus V_2$ definieren wir die
\underline{direkte Summe von Darstellungen}
\index{direkte Summe von Darstellungen} $\rho = 
\rho_1 \oplus \rho_2$ durch komponentenweise Wirkung. 
Analog definiert man dies für beliebige Indexfamilien
$I$.
\end{definition}

Besonders relevant werden auch Beziehungen zwischen
Darstellungen, die wie üblich über Abbildungen gegeben sind.
\begin{definition}[ $G$-Homomorphismen, Äquivalenz von 
    Darstellungen]
    Seien $( \rho, V_ \rho), ( \pi, V_ \pi)$ zwei
    Darstellungen von $G$.
    \begin{enumerate}
        \item Ein stetiger, linearer Operator $T: V_ \rho 
            \to V_ \pi$ heißt \underline{$G$-Homomorphismus}
            \index{$G$-Homomorphismus}, falls
            $$ T \circ \rho(g) = \pi(g) \circ T$$
            für alle $ g \in G$ gilt.
        \item Die Menge aller $G$-Homomorphismen von 
            $V_ \rho$ nach $ V_ \pi$ bezeichnen wir mit
            $\Hom_G(V_ \rho, V_ \pi)$.
        \item Falls $ \rho$ und $ \pi$ unitäre Darstellungen
            sind, heißen diese \underline{unitär äquivalent}
            \index{unitär äquivalent}, falls es einen
            unitären $G$-\\Homomorphismus $T: V_ \rho\to 
            V_ \pi$ gibt.
    \end{enumerate}
\end{definition}
\section{Lemma von Schur}
Wie man sich schon vorstellen kann, ist die Theorie
irreduzibler Darstellungen sehr überschaubar. Ein
zentrales Resultat dieser ist das sogenannte Lemma
von Schur:
\begin{lemma}[Schur]\label{lem:Schur}
Seien $( \rho, V_ \rho)$ eine unitäre Darstellung einer 
topologischen Gruppe $G$. Dann sind äquivalent:
\begin{enumerate}
    \item $ \rho$ ist irreduzibel.
    \item Falls $T \in \Hom_G( V_ \rho, V_ \rho)$, dann
        gilt $ T \in \C \id$.
\end{enumerate}
\end{lemma}
Für den Beweis dieses Lemma brauchen wir noch ein wenig 
Vorarbeit aus der Funktionalanalysis:
\begin{lemma}\label{lem:topologicalSchur}
Sei $H$ ein Hilbertraum und $A \subseteq \mathcal L(H)$ eine
Menge beschränkter Operatoren auf $H$, sodass für alle $S 
\in A$ auch $S^* \in A$ gilt. Dann sind äquivalent:
\begin{enumerate}
    \item Die einzige $A$-invariante, abgeschlossene Menge
        ungleich $ \{ 0 \} $ ist $ V$.
    \item Falls $T \in \mathcal L(H)$ mit allen $S \in A$
        kommutiert, dann ist $T = \lambda \id_V$ für
        ein $ \lambda \in \C$.
\end{enumerate}

\end{lemma}
\begin{proof}
Angenommen, (2) gilt. Dann sei $ \{ 0 \} \neq L \subseteq H$
ein abgeschlossener, $A$-invarianter Unterraum von $H$. Dann
ist auch $ L^\perp$ ein $A$-invarianter, abgeschlossener 
Untervektorraum, denn für alle $T \in A$, $x \in L^\perp$
und $y \in L$ gilt
$$ \langle Tx, y\rangle = \langle x, T^*y \rangle.$$
Da $T^* \in A$, folgt $T^* y \in L$. Somit gilt auch 
$$\langle x, T^*y \rangle = 0.$$
Da die gewählten Elemente beliebig waren, folgt $Tx \in 
L^\perp$ und somit ist $L^\perp$ ein $A$-invarianter,
abgeschlossener Untervektorraum. Dann betrachten wir
die orthogonale Projektion auf $L$ $P_L: H\to L$. Aus
obiger Rechnung sehen wir direkt mittels $H = L\oplus 
L^\perp$, dass diese mit allen $T \in A$ kommutiert.
Unter Ausnutzung von (2) erhalten wir also, dass 
$P_L = \id$ (da Projektionen Operatornorm 1 haben) gilt.
Somit muss $L= H$ gelten.\\
Angeommen (1) gilt. Sei also $T \in \mathcal L(H)$ ein 
Operator, der mit $A$ kommutiert. Dann kommutiert auch $T^*$
mit $A$, denn für $S \in A$ ist $S^* \in A$ ud somit
\begin{align*}
    \langle S T^* x, y\rangle &= \langle T^* x, S^* y
    \rangle\\
                                   &= \langle x, T S^*
                                      y\rangle\\
                                   &= \langle x, S^*T
                                   \rangle \\
                                   &= \langle T^* Sx, y
                                   \rangle,
\end{align*}
da $S^* \in A$. Da die Matrixkoeffizienten übereinstimmen,
folgt $S T^* = T^* S$. Indem wir $T= \frac{1}{2}(T+T^*) -
\frac{1}{2}i(iT - iT^*)$ betrachten, können wir ohne
Einschränkung annehmen, dass $T$ selbstadjungiert ist.
Da die Aussage für $T=0$ mit $ \lambda = 0$ klar ist,
können wir zudem $T\neq 0$ annehmen. Wir zeigen, dass
das Spektrum von $T$ aus einem einzigen Punkt
besteht. \\
Angenommen, es gibt $x\neq y \in \sigma(T)$. Dann gibt
es zwei Funktionen $f,g\in C( \sigma(T))$ mit
$f(x)\neq 0 \neq g(y)$ und $f\cdot g = 0$. Dann betrachten
wir das stetige Funktionalkalkül $f(T)$ und $g(T)$ und 
erinnern uns daran, dass jeder Operator $S$, der mit $T$ 
kommutiert, auch mit $f(T)$ kommutiert für 
$f\in C(\sigma(T))$.
Dann ist $f(T)\neq 0 \neq g(T)$ und $f\cdot g (T) = 0$,
da das holomorphe Funktionalkalkül multiplikativ ist.\\
Da $g(T)$ mit $A$ kommutiert, ist $L = \closure{g(T)H}$
ein $A$-invarianer Untervektorraum von $H$ ungleich dem 
Nullraum. Aus Anwendung von (1) erhalten wir, dass $L=H$.
Aber es gilt gleichzeitig auch, dass $ \{ 0 \} \neq f(T)H
= f(T) L \subseteq \closure{f(T) g(T) H} = \{ 0 \} $,
was einen Widerspruch darstellt. Somit muss das Spektrum
einelementig sein.
\end{proof}
Zusätzlich brauchen wir noch ein kleines Lemma, um ein
wenig mehr über unitäre Darstellungen zu erfahren.
\begin{lemma}\label{lem:unitaryReprMapsInversesToAdjoints}
Sei $( \rho, V)$ eine Darstellung. Dann sind äquivalent:
\begin{enumerate}
    \item $ \rho$ ist unitär.
    \item $ \rho( g^{-1}) = \rho(g)^*$ für alle $g \in G$.
\end{enumerate}
\end{lemma}
\begin{proof}
    \begin{align*}
        \rho \text{ ist unitär} &\iff \rho(g)^* \text{ 
        unitär }\forall g \in G\\
                                &\iff \rho(g)^{-1} = 
                                \rho(g)^* \forall g \in G\\
                                &\iff \rho( g^{-1}) = 
                                \rho(g)^* \forall g \in G,
    \end{align*}
    wobei im letzten Schritt verwendet wurde, dass $ \rho$
    ein Gruppenhomomorphismus ist.
\end{proof}
Nun können wir das Lemma \hyperref[lem:Schur]{Lemma von Schur} 
beweisen, welches mehr oder weniger ein direktes Korollar 
aus den obigen Lemmata \ref{lem:topologicalSchur} und
\ref{lem:unitaryReprMapsInversesToAdjoints} ist.
\begin{proof}[Beweis (Schur)]
    Nach Lemma \ref{lem:unitaryReprMapsInversesToAdjoints}
    ist $ A\coloneq\{ \rho(g): g \in G \} $ eine Menge, 
    für die $ S \in A \iff S^* \in A$ gilt. Daher können
    wir Lemma \ref{lem:topologicalSchur} anwenden und 
    erhalten die gewünschte Aussage.
\end{proof}
Während das Lemma von Schur im ersten Moment so wirkt,
als könne man damit nur die Homomorphismen einer Darstellung
auf sich selbst klassifizieren, liefert es auch viele
Informationen für die Homomorphismen zwischen beliebigen
irreduziblen Darstellungen.
\begin{korollar}\label{cor:HomSpaceOneDimensional}
Seien $( \rho, V_ \rho), ( \pi, V_ \pi)$ zwei irreduzible,
unitäre Darstellungen. Dann gilt für einen $G$-
Homomorphismus $T: V_ \rho\to V_ \pi$, dass $T=0$ oder dass
$T$ invertierbar mit stetiger Umkehrabbildung ist. In diesem
Fall existiert ein $c>0$, sodass $ cT$ unitär ist. Dies
zeigt insbesondere, dass $\Hom_G(V_ \rho, V_ \pi) = \{ 0 \}$
gilt, außer $ \rho$ und $ \pi$ sind unitär äquivalent.
In diesem Fall ist $\dim(\Hom_G( V_ \rho, V_ \pi)) = 1$-
\end{korollar}
\begin{proof}
Sei $T: V_ \rho \to V_ \pi$ ein $G$-Homomorphismus. 
Betrachte $T^*: V_ \pi \to V_ \rho$. Diese Abbildung ist
ebenfalls ein $G$-Homomorphismus wie folgende Rechnung
zeigt:
\begin{align*}
    \langle \rho(g) T^* x, y \rangle &= \langle T^*x, 
    \rho(g)^* y \rangle\\ 
                                     &= \langle x, T \rho( 
                                     g^{-1}) y \rangle \\
                                     &= \langle x, 
                                 \pi(g^{-1})Ty\rangle\\
                                     &= \langle \pi(g)x, 
                                     Ty \rangle \\
                                     &= \langle T^* \pi(g)x
                                     ,y \rangle, 
\end{align*}
für alle $g \in G, x,y \in H$. Damit ist $T^* T \in 
\Hom_G(V_ \rho, V_ \rho)$. Aus dem \hyperref[lem:Schur]{
Lemma von Schur} folgt, dass $T^* T = \lambda \id$ für
ein $ \lambda \in \C$. Falls $T\neq 0$, dann ist
$T^* T \neq 0$. Durch die positive Semidefinitheit der 
Abbildung, folgt $ \lambda >0$. Betrachte $c = 
\sqrt{ \lambda^{-1}}$. Dann ist $(cT)^*(cT) = \id$. 
Zusätzlich folgt analog, dass $T T^*$ bijektiv ist. Damit
muss $cT$ bijektiv sein und somit auch unitär.\\
Für je zwei $G$-Homomorphismen $S,T: V_ \rho\to V_ \pi$ mit
$T\neq 0$ folgt also, dass $S \circ T^{-1} \in \Hom_G(V_ 
\pi, V_ \pi)$. Somit $S \circ T^{-1} = \lambda \id$ für
ein $ \lambda \in\C$. Somit $S = \lambda T$. Dies zeigt
die Eindimensionalität.
\end{proof}
\section{Kompakte Gruppen}
Das Lemma von Schur liefert uns schon eine gute
Charakterisierung irreduzibler Darstellungen. Jedoch
interessieren uns auch nicht notwendigerweise irreduzible
Darstellungen. Diese können im Allgemeinen etwas
unübersichtlicher werden. Für kompakte Gruppen wird die 
Situation jedoch recht überschaubar und hier wird uns das
erste Mal die Integration bezüglich des Haarmaßes gute 
Dienste erweisen. \\
Zuerst beginnen wir damit, die endlichdimensionalen 
Darstellungen genauer zu studieren. Dies wirkt auf den 
ersten Blick stark einschränkend. Wir werden jedoch sehen,
wie mächtig diese Darstellungen auf kompakten Gruppen 
sind.\\
Bevor wir damit jedoch starten können, müssen wir zumindest
kurz über die Haarmaße auf kompakten Gruppen reden. A
priori ist nämlich nicht klar, ob es einen Unterschied 
zwischen linken und rechten Haarmaßen gibt oder nicht.
Dazu brauchen wir einige Hilfsmittel.
\begin{definition}[modulare Funktion, unimodulare Gruppen]
    Sei $G$ eine lokalkompakte Gruppe und $ \mu$ ein
    Haarmaß auf $G$. Für $x \in G$ definieren wir das Maß
    $ \mu_x$ durch $ \mu_x(A) = \mu(Ax)$, wobei es sich
    ebenfalls um ein Haarmaß handelt. Somit gibt es durch
    die Eindeutigkeit des Haarmaßes eine Zahl $\Delta(x)>0$,
    mit $ \mu_x = \Delta(x) \mu$. Dies definiert eine 
    Abbildung $ \Delta: G\to \R_{>0}$, welche die
    \underline{modulare Funktion}\index{modulare Funktion}
    der Gruppe $G$ genannt wird.\\
    Wenn $ \Delta \equiv 1$, dann heißt $G$ eine
    \underline{unimodulare Gruppe}\index{unimodulare Gruppe}
    .
\end{definition}
\begin{bemerkung}
Für unimodulare Gruppen stimmen linke und rechte Haarmaße
überein.
\end{bemerkung}
Damit wir uns keine Gedanken über das Maß machen müssen,
wollen wir zeigen, dass kompakte Gruppen unimodular sind.
\begin{satz}\label{thm:propsModularFunction}
Sei $G$ eine lokalkompakte Gruppe mit modularer Funktion
$ \Delta: G\to \R^\times_{>0}$. Dann gelten folgende 
Eigenschaften:
\begin{enumerate}
    \item Für $ y \in G$ und $ f \in \LL^1(G)$ ist
        $R_y f \in \LL^1(G)$ und es gilt
        $$ \int_{G} R_y f(x) \,\diff x = \int_{G} f(xy) 
        \,\diff x = \Delta( y^{-1}) \int_{G} f(x) 
        \,\diff x,$$
        wobei $R_yf: G\to \R, x \mapsto f(xy)$.
    \item $ \Delta$ ist ein stetiger Gruppenhomomorphismus.
    \item Kompakte Gruppen sind unimodular.
\end{enumerate}
\end{satz}
\begin{proof}
    \begin{enumerate}
        \item Für charakteristische Funktionen 
            $\mathds{1}_A$ gilt:
            \begin{align*}
                \int_{_G} \mathds{1}_A(xy) \,\diff x &=
                \int_{A} xy \,\diff x \\
                &= \int_{Ay^{-1}} x \,\diff  x\\
                &= \mu(A y^{-1})\\
                &= \Delta( y^{-1}) \mu(A)\\
                &= \Delta( y^{-1}) \int_{G} \mathds{1}_A(x)
                \,\diff x.
            \end{align*}
            Für beliebige integrierbare Funktionen folgt die
            Aussage mit maßtheoretischer Induktion.
        \item Für $x,y \in G$ und eine messbare Menge
            $A \subseteq G$ gilt
            \begin{align*}
                \Delta(xy) \mu(A) &= \mu_{xy}(A) \\
                                  &= \mu(Axy)\\
                                  &= \mu_y(Ax)\\
                                  &= \Delta(y) \mu(Ax)\\
                                  &= \Delta(y) \Delta(x) 
                                  \mu(A).
            \end{align*}
            Durch Wahl einer beliebigen Menge $A$ mit
            $ 0 < \mu(A) < \infty$ gilt dann
            $$ \Delta(xy) = \Delta(x) \Delta(y).$$
            Zur Stetigkeit in $y$ wähle eine Funktion
            $f \in C_c(G)$ mit $ c = \int_{G} f(x)  \,\diff
            x > \neq 0$. Dann folgt durch den ersten Teil
            der Aussage 
            $$ \Delta(y) = \frac{1}{c} \int_{G} f(x y^{-1}) 
            \,\diff x  = \frac{1}{c} \int_{G} R_{ y^{-1}} 
            f(x) \,\diff  x.$$
            Dieser Ausdruck ist stetig in $y$, weswegen auch
            $ \Delta$ stetig in $y$ sein muss.
        \item Sei nun $G$ kompakt. Da $ \Delta$ ein
            stetiger Gruppenhomomorphismus ist, muss
            $ \Delta(G)$ eine kompakte Untergruppe von
            $ \R_{>0}^\times$ sein. Die einzige kompakte
            Untergruppe dieser Gruppe ist jedoch $ \{ 1 \}$.
    \end{enumerate} 
\end{proof}
Dass wir nicht zwischen linken und rechten Haarmaßen
unterscheiden müssen, erleichtert die Situation erheblich,
weswegen wir uns nun den endlichdimensionalen Darstellungen
widmen können. Sei also im Folgenden stets $K$ eine kompakte
(also insbesondere lokalkompakte) topologische Gruppe sowie
$ (\rho, V)$ eine endlichdimensionale Darstellung, sofern
nicht anders gefordert.
\begin{lemma}
\label{lem:finiteDimensionalRepresentationsAlwaysUnitary}
Auf $V$ gibt es stets ein Skalarprodukt, sodass $ \rho$
eine unitäre Darstellung wird. Falls $ \rho$ irreduzibel
ist, dann ist dieses Skalarprodukt eindeutig bis auf
eine positive Konstante bestimmt.
\end{lemma}
\begin{proof}
Wir wählen ein beliebiges Skalarprodukt $( \cdot, \cdot): 
V\times V\to \C$. Dann definieren wir ein neues 
Skalarprodukt durch den Ausdruck
$$ \langle v, w \rangle \coloneq \int_{K} (
 \rho(k)v, \rho(k)w)\,\diff k$$
bezüglich des Haarmaßes $ \mu$, für das $ \mu(K) = 1$ gilt.
Wir zeigen nun, dass dies tatsächlich ein Skalarprodukt
definiert:
\begin{enumerate}
    \item Die Linearität folgt direkt aus der Linearität
        von $ (\cdot, \cdot)$ und des Integrals.
    \item Dass das Skalarprodukt hermitesch ist, folgt
        unmittelbar aus der Hermitizität des ursprünglichen
        Skalarproduktes und dem Fakt, dass Konjugieren und
        Integrieren kommutieren.
    \item Es bleibt nur noch zu zeigen, dass die so
        definierte Abbildung positiv definit ist. Dabei ist
        die Positivität ebenfalls klar. Sei also $ v \in V$,
        sodass $ \langle v,v \rangle =0$. Dann gilt
        $$ 0 = \langle v,v \rangle = \int_{K} ( \tau(k)v, 
        \tau(k)v)  \,\diff k.$$
        Die Abbbildung $ k \mapsto (\tau(k)v, 
        \tau(k)v) $ ist stetig und somit können
        wir obiges Korollar aus der Maßtheorie %TODO 
        verwenden und erhalten, dass $ k \mapsto ( \tau(k)v
        , \tau(k)v)$ gleich der Nullabbildung ist. 
        Insbesondere gilt also für $k = 1_G$, dass
        $ \tau(1_G) = \id_V$. Somit folgt also ebenfalls
        $ (v, v) = 0$. Da $(\cdot, \cdot)$ ein Skalarprodukt
        ist, folgt $ v = 0$.
\end{enumerate}
Damit ist $ \langle \cdot, \cdot \rangle $ ein Skalarprodukt
und wir müssen nur noch zeigen, dass $ \rho$ unitär 
bezüglich diesen Skalarprodukts ist. Dazu sehen wir, dass
\begin{align*}
    \langle \tau(x)v, \tau(x)w \rangle 
    &= \int_{K}(\tau(k)\tau(x)v, \tau(k)\tau(x)w)\,\diff k\\
    &= \int_{K}( \tau(kx)v, \tau(kx)w) \,\diff  k\\
    &= \int_{K} ( \tau(k)v, \tau(k)w) \,\diff k\\
    &= \langle v,w \rangle, 
\end{align*}
da die Gruppe $K$ unimodular ist und es sich mit $f:
K\to \C, k\mapsto ( \tau(k)v, \tau(k)w)$ bei
$R_x f : K\to \C$ genau um $k \mapsto ( \tau(kx)v, 
\tau(kx)w)$ handelt und wir somit 
Satz \ref{thm:propsModularFunction} anwenden können. Was
noch zu zeigen ist, ist die Eindeutigkeit bis auf 
positive Konstanten im Fall irreduzibler Darstellungen. 
Dazu seien $( \rho_1, V_1), (\rho_2, V_2)$ Darstellungen,
die sich nur im unitären Skalarprodukt auf $V$ 
unterscheiden. Bezeichne diese Skalarprodukte mit 
$ \langle \cdot, \cdot \rangle_1$ und 
$ \langle \cdot,\cdot \rangle_2$. Betrachte nun die
lineare Abbildung $\id:V_1\to V_2, v\mapsto v$. Da
$V_1$ und $V_2$ endlichdimensional ist, ist $\id$ 
stetig und ein $G$-Homomorphismus zwischen $ \rho_1$
und $\rho_2$. Nun nutzen wir Korollar
\ref{cor:HomSpaceOneDimensional}, um zu folgern, dass
es ein $c>0$ gibt, sodass $c\cdot \id$ unitär zwischen
den beiden Darstellungen ist. Insbesondere folgt dann
also, dass 
\begin{align*}
	c^2 \langle v_1, v_2 \rangle_2 
	&= \langle c v_1, c v_2 \rangle_2 \\
	&= \langle v_1, v_2 \rangle_1
\end{align*}
Damit sind die Skalarprodukte bis auf positive Konstante
gleich.
\end{proof}
Zusätzlich lassen sich die endlichdimensionalen 
Darstellungen gut in "Atome" -- die irreduziblen 
Darstellungen -- zerlegen.
\begin{lemma}\label{lem:FiniteDimReprSumOfIrredRepr}
Sei $( \rho, V)$ eine endlichdimensionale Darstellung 
von $K$. Dann gibt es irreduzible Darstellungen 
$(\rho_i, V_i)_{i \in I}$, sodass $$ \rho = 
\bigoplus_ { i \in I } \rho_i$$.
\end{lemma}
\begin{proof}
Sei $K$ eine kompakte, topologische Gruppe und $(\rho, V)$
eine Darstellung von $K$. Da $V$ endlichdimensional ist,
können wir die Aussage mit Induktion über $\dim(V)$
führen.\\
\underline{I.A. ($\dim(V)=1$):} Dieser Fall ist klar, da die
einzigen Unterräume von $V$ $ \{ 0 \} $ und $V$ sind.
Somit muss jede eindimensionale Darstellung schon
direkt irreduzibel sein.\\
\underline{I.S.:} Sei $ V$ ein Vektorraum der Dimension
$ n+1$ für ein beliebiges $ n \in \N_0$. Angenommen, wir 
haben die Aussage schon für alle Vektorräume der 
Dimension kleiner gleich $n$ gezeigt. Zusätzlich können
wir mit Lemma 
\ref{lem:finiteDimensionalRepresentationsAlwaysUnitary}
annehmen, dass $ \rho$ eine unitäre Darstellung ist. 
Falls $\rho$ bereits irreduzbiel ist, sind wir fertig.
Ansonsten gibt es einen $\rho(K)$-invarianten 
Untervektorraum $ \{ 0 \} \subsetneq  U \subsetneq V$. 
Betrachte dessen orthogonales Komplement $U^\perp$. 
Auch dieser ist abgeschlossen unter $\rho(K)$, denn 
für $ g \in K, u \in U, v \in U^\perp$ gilt:
\begin{align*}
	\langle u, \rho(g)v \rangle 
	&= \langle \rho(g)^*u, v \rangle \\
	&= \langle \rho(g^{-1})u, v \rangle\\
	&= 0,
\end{align*}
wobei wir im vorletzten Schritt verwendet haben,
dass $ \rho(g)$ unitär ist, und im letzten
Schritt, dass $U$ $\rho(K)$-invariant ist.
Nun können wir die Induktionsvoraussetzung auf
$U$ und $U^\perp$ anwenden und erhalten durch
$ V = U \oplus U^\perp$ die gewünschte Aussage.
\end{proof}
Wir beenden diesen Abschnitt mit zwei Definitionen
und einem Korollar:
\begin{definition}[unitäres Dual, endliches unitäres Dual]
Sei $G$ eine lokalkompakte Gruppe. Dann bezeichnen
wir mit $\UDual{G}$ die Menge aller Äquivalenzklassen
irreduzibler Darstellungen auf $G$ bezüglich
unitärer Äquivalenz. Wir nennen $\UDual{G}$
\underline{unitäres Dual}\index{unitäres Dual}
zu $G$.\\
Zusätzlich bezeichnen wir für eine kompakte Gruppe $K$
die Äquivalenzklassen der endlichdimensionalen,
irreduziblen Darstellungen mit $\UDualFin{K}$ und
nennen dies das \underline{endliche, unitäre Dual}
\index{endliche, unitäre Dual} von $K$.
Darstellungen
\end{definition}
\begin{bemerkung}
A priori gibt es ein kleines mengentheoretisches Problem
bei der Definition dieser beiden Mengen, da nicht trivial
ist, dass es sich wirklich um Mengen handelt. Um dies zu
beheben, müssten wir ein wenig in die Theorie der 
Kardinalzahlen eintauchen. Dies wollen wir hier nicht
weiter vertiefen.
\end{bemerkung}
Ein wesentliches Ziel des Vortrags ist es zu zeigen,
dass $\UDual{K} = \UDualFin{K}$ gilt.
\begin{korollar}
	Es gilt stets $\UDualFin{K}\subseteq \UDual{K}$.
\end{korollar}
\begin{proof}
Zu zeigen ist bloß, dass alle endlichdimensionalen
Darstellungen auch unitär sind. Dies haben wir 
bereits in Lemma 
\ref{lem:finiteDimensionalRepresentationsAlwaysUnitary}
gezeigt.
\end{proof}
\chapter{Der Satz von Peter-Weyl}
\section{Orthogonalitätseigenschaften}

\printindex
\end{document}
