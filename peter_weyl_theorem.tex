\documentclass{report}

\usepackage{style}
\addbibresource{references.bib}

\begin{document}
\tableofcontents
\chapter{Einführung}
\section{Motivation}
Das zentrale Ergebnis des letzten Vortrags war die Existenz
und Eindeutigkeit des Haarmaßes, welche sich im folgenden
Satz zusammenfassen lässt.
\begin{satz*}[Existenz und Eindeutigkeit des Haarmaßes]
    Sei $G$ eine lokalkompakte Gruppe. Dann gibt es eine 
    linksinvariantes Radonmaß $ \mu\neq 0$ auf 
    $\mathcal{B}(X)$. Dieses ist eindeutig bis auf 
    positive Vielfache.
\end{satz*}
Dieses Maß wird als das Haarmaß bezeichnet. Somit erhalten
wir auf jeder Gruppe einen sinnvollen Begriff des Integrals.
Dies erlaubt uns, wirklich sinnvoll über Räume wie 
$\LL^2(G)$ zu reden.\\
Im Spezialfall erhalten wir dann natürlich Ergebnisse für
die Integrationstheorie auf dem $\R^d$ oder anderen 
spannenden Gruppen wie $\R^\times$. Jedoch sind nicht alle
Gruppen so gut strukturiert wie diese Gruppen. Diese sind
nämlich abelsch und lokalkompakt. Die Integrationstheorie
auf solchen Gruppen ist zentraler Gegenstand der 
\emph{harmonischen Analysis}. Der nicht abelsche Fall ist
jedoch nicht so gut strukturiert und deutlich 
unübersichtlicher. Für den kompakten Fall erhalten wir 
jedoch den \emph{Satz von Peter-Weyl}, welcher viele 
Informationen über die Struktur von $\LL^2(G)$ beinhaltet.
Diesen zu verstehen und zu beweisen ist Gegenstand des 
vorliegenden Seminarvortrags.\\
Zunächst brauchen wir jedoch noch Grundlagen der
Darstellungstheorie topologischer Gruppen. Die Primärquelle
dieses Vortrags ist \cite{principlesHarmAna}. An einigen 
Stellen greifen wir auf \cite{reprCompLieGroups} zurück.
\section{Notation}
\begin{itemize}
    \item Für einen Banachraum $V$ bezeichnen wir mit 
        $\GLS{V}$ die Menge der stetigen, linearen, 
        invertierbaren Abbildungen.
    \item Für zwei Banachräume $V, W$ bezeichnen wir mit
        $ \mathcal L(V, W)$ die Menge der beschränkten,
        stetigen linearen Abbildungen $T:V\to W$.
    \item Für einen Maßraum $( \Omega, \mathcal{A}, \mu)$
	    bezeichnen wir mit $\LL^2( \Omega)$ den
	    üblichen Raum der quadratintegrierbaren
	    Funktionen $f: \Omega\to \C$. Auf diesem
	    ist das übliche Skalarprodukt gegeben.
	    Dieses wird mit $ \langle \cdot, 
	    \cdot \rangle_{\LL^2}$ notiert.
\end{itemize}
\chapter{Grundlagen}
\section{Haarsche Integrationstheorie}
\section{Funktionalanylsis}
\chapter{Darstellungstheorie}
Ziel der Darstellungstheorie ist es gewissermaßen, das 
Problem der schwierigen Strukturierung von Gruppen zu
lösen, indem man die Elemente in lineare Operatoren 
übersetzt.
Für diese kennen wir nämlich aus der linearen Algebra im 
endlichdimensionalen sowie der Funktionalanalysis um 
unendlichdimensionalen Fall viele hilfreiche Sätze, mit 
denen wir sehr häufig mehr über die Struktur der Gruppe 
verstehen können.
\section{Grundlegende Begriffe}
Der zentrale Begriff in der Darstellungstheorie 
topologischer Gruppen ist der einer Darstellung.
\begin{definition}[Darstellung]
Sei $V$ ein Banachraum und $G$ eine topologische Gruppe. 
Eine \underline{Banachraum-Darstellung}
\index{Banachraum-Darstellung}\footnote{Häufig werden wir 
nur Darstellung sagen und damit eine Banachraum-Darstellung 
meinen.} von $G$ auf $V$ ist ein Gruppenhomomorphismus 
$ \pi: G\to \GLS{V}$, sodass die Abbildung $$G\times V\to V,
(g,v)\mapsto (g,v)\mapsto \pi(g)v$$ stetig ist.
\end{definition}
\begin{beispiel}[Linksreguläre Darstellung]
    \label{ex:leftRegularRepr}
Jede lokalkompakte Gruppe $G$ wirkt auf dem Hilbertraum
ihrer quadratintegrierbaren Funktionen $ \LL^2(G)$ mit
der sogenannten \underline{linksregulären Wirkung}
\index{linksreguläre Wirkung}, welche durch 
$$ \rho: G\to \GLS{\LL^2(G)}, x\mapsto L_x$$
gegeben wird mit $ L_x: \LL^2(G)\to \LL^2(G), \phi 
\mapsto \phi \circ l_{ x^{-1}}$, wobei $l_x$ die 
Linksmultiplikation mit einem Element $x$ bezeichnet.
\end{beispiel}

Bereits aus der Algebra 1 wissen wir um die Wichtigkeit 
bestimmter "elementarer" Strukturen, z.B. den zyklischen
Gruppen bei der Klassifikation endlicher abelscher Gruppen.
Auch in der Darstellungstheorie gibt es einen ähnlichen
Begriff:
\begin{definition}[Unterdarstellungen, irreduzible 
    Darstellungen]
Sei $( \rho, V_ \rho)$ eine Darstellung einer topologischen
    Gruppe $G$. \begin{enumerate}
    \item $( \pi, V_ \pi)$ heißt \underline{
        Unterdarstellung}\index{Unterdarstellung} von 
        $ ( \rho, V_ \rho)$, falls $ V_ \pi$ ein 
        abgeschlossener Untervektorraum von $ V_ \rho$
        ist und $ \rho$ eingeschränkt auf $ V_ \pi$
        gleich $ \pi$ ist.
    \item $( \rho, V_ \rho)$ heißt \underline{irreduzibel}
        \index{irreduzibel}, falls es keine nicht-trivialen
        Unterdarstellungen gibt, also falls für jede 
        Unterdarstellung $( \pi, V_ \pi)$ gilt, dass
        $V_ \pi = \{0\}$ oder $ V_ \pi = V_ \rho$.
\end{enumerate}
\end{definition}
\begin{bemerkung}
Jeder abgeschlossene Untervektorraum $U \subseteq V_ \rho$, 
für den $ \rho(G) U \subseteq U$ gilt, liefert eine 
Unterdarstellung durch Einschränken von $ \rho$ auf $ U$.
\end{bemerkung}

Für den Fall, dass die Banachräume $V$ für unsere 
Darstellungen sogar Hilberträume sind, gibt es noch
eine weitere Art von Darstellungen:
\begin{definition}[Unitäre Darstellungen]
Sei $V$ ein Hilbertraum und $G$ eine topologische Gruppe.
Eine Darstellung $ \rho$ auf $ V$ heißt \underline{unitäre 
Darstellung}\index{unitäre Darstellung}, falls 
$ \rho(g)$ unitär ist für alle $g \in G$, also
$ \langle \rho(g)v, \rho(g)w \rangle = \langle v, w\rangle$
für alle $ g \in G$ und alle $ v,w \in V$.
\end{definition}
Wir möchten später zeigen, dass sich jede endlichdimensionale
Darstellung als direkte Summe von irreduziblen Darstellungen
schreiben lässt. Dazu definieren wir
\begin{definition}[Direkte Summe von Darstellungen]
Seien $( \rho_1, V_1), (\rho_2, V_2)$ zwei unitäre 
Darstellungen. Auf der direkten Summe von Vektorräumen
$V \coloneq V_1 \oplus V_2$ definieren wir die
\underline{direkte Summe von Darstellungen}
\index{direkte Summe von Darstellungen} $\rho = 
\rho_1 \oplus \rho_2$ durch komponentenweise Wirkung. 
Analog definiert man dies für beliebige Indexfamilien
$I$.
\end{definition}

Besonders relevant werden auch Beziehungen zwischen
Darstellungen, die wie üblich über Abbildungen gegeben sind.
\begin{definition}[ $G$-Homomorphismen, Äquivalenz von 
    Darstellungen]
    Seien $( \rho, V_ \rho), ( \pi, V_ \pi)$ zwei
    Darstellungen von $G$.
    \begin{enumerate}
        \item Ein stetiger, linearer Operator $T: V_ \rho 
            \to V_ \pi$ heißt \underline{$G$-Homomorphismus}
            \index{$G$-Homomorphismus}, falls
            $$ T \circ \rho(g) = \pi(g) \circ T$$
            für alle $ g \in G$ gilt.
        \item Die Menge aller $G$-Homomorphismen von 
            $V_ \rho$ nach $ V_ \pi$ bezeichnen wir mit
            $\Hom_G(V_ \rho, V_ \pi)$.
        \item Falls $ \rho$ und $ \pi$ unitäre Darstellungen
            sind, heißen diese \underline{unitär äquivalent}
            \index{unitär äquivalent}, falls es einen
            unitären $G$-\\Homomorphismus $T: V_ \rho\to 
            V_ \pi$ gibt.
    \end{enumerate}
\end{definition}
\section{Lemma von Schur}
Wie man sich schon vorstellen kann, ist die Theorie
irreduzibler Darstellungen sehr überschaubar. Ein
zentrales Resultat dieser ist das sogenannte Lemma
von Schur:
\begin{lemma}[Schur]\label{lem:Schur}
Seien $( \rho, V_ \rho)$ eine unitäre Darstellung einer 
topologischen Gruppe $G$. Dann sind äquivalent:
\begin{enumerate}
    \item $ \rho$ ist irreduzibel.
    \item Falls $T \in \Hom_G( V_ \rho, V_ \rho)$, dann
        gilt $ T \in \C \id$.
\end{enumerate}
\end{lemma}
Für den Beweis dieses Lemma brauchen wir noch ein wenig 
Vorarbeit aus der Funktionalanalysis:
\begin{lemma}\label{lem:topologicalSchur}
Sei $H$ ein Hilbertraum und $A \subseteq \mathcal L(H)$ eine
Menge beschränkter Operatoren auf $H$, sodass für alle $S 
\in A$ auch $S^* \in A$ gilt. Dann sind äquivalent:
\begin{enumerate}
    \item Die einzige $A$-invariante, abgeschlossene Menge
        ungleich $ \{ 0 \} $ ist $ H$.
    \item Falls $T \in \mathcal L(H)$ mit allen $S \in A$
        kommutiert, dann ist $T = \lambda \id_V$ für
        ein $ \lambda \in \C$.
\end{enumerate}

\end{lemma}
\begin{proof}
Angenommen, (2) gilt. Dann sei $ \{ 0 \} \neq L \subseteq H$
ein abgeschlossener, $A$-invarianter Unterraum von $H$. Dann
ist auch $ L^\perp$ ein $A$-invarianter, abgeschlossener 
Untervektorraum, denn für alle $T \in A$, $x \in L^\perp$
und $y \in L$ gilt
$$ \langle Tx, y\rangle = \langle x, T^*y \rangle.$$
Da $T^* \in A$, folgt $T^* y \in L$. Somit gilt auch 
$$\langle x, T^*y \rangle = 0.$$
Da die gewählten Elemente beliebig waren, folgt $Tx \in 
L^\perp$ und somit ist $L^\perp$ ein $A$-invarianter,
abgeschlossener Untervektorraum. Dann betrachten wir
die orthogonale Projektion auf $L$ $P_L: H\to L$. Aus
obiger Rechnung sehen wir direkt mittels $H = L\oplus 
L^\perp$, dass diese mit allen $T \in A$ kommutiert.
Unter Ausnutzung von (2) erhalten wir also, dass 
$P_L = \id$ (da Projektionen Operatornorm 1 haben) gilt.
Somit muss $L= H$ gelten.\\
Angeommen (1) gilt. Sei also $T \in \mathcal L(H)$ ein 
Operator, der mit $A$ kommutiert. Dann kommutiert auch $T^*$
mit $A$, denn für $S \in A$ ist $S^* \in A$ ud somit
\begin{align*}
    \langle S T^* x, y\rangle &= \langle T^* x, S^* y
    \rangle\\
                                   &= \langle x, T S^*
                                      y\rangle\\
                                   &= \langle x, S^*T
                                   \rangle \\
                                   &= \langle T^* Sx, y
                                   \rangle,
\end{align*}
da $S^* \in A$. Da die Matrixkoeffizienten übereinstimmen,
folgt $S T^* = T^* S$. Indem wir $T= \frac{1}{2}(T+T^*) -
\frac{1}{2}i(iT - iT^*)$ betrachten, können wir ohne
Einschränkung annehmen, dass $T$ selbstadjungiert ist.
Da die Aussage für $T=0$ mit $ \lambda = 0$ klar ist,
können wir zudem $T\neq 0$ annehmen. Wir zeigen, dass
das Spektrum von $T$ aus einem einzigen Punkt
besteht. \\
Angenommen, es gibt $x\neq y \in \sigma(T)$. Dann gibt
es zwei Funktionen $f,g\in C( \sigma(T))$ mit
$f(x)\neq 0 \neq g(y)$ und $f\cdot g = 0$. Dann betrachten
wir das stetige Funktionalkalkül $f(T)$ und $g(T)$ und 
erinnern uns daran, dass jeder Operator $S$, der mit $T$ 
kommutiert, auch mit $f(T)$ kommutiert für 
$f\in C(\sigma(T))$.
Dann ist $f(T)\neq 0 \neq g(T)$ und $f\cdot g (T) = 0$,
da das holomorphe Funktionalkalkül multiplikativ ist.\\
Da $g(T)$ mit $A$ kommutiert, ist $L = \closure{g(T)H}$
ein $A$-invarianer Untervektorraum von $H$ ungleich dem 
Nullraum. Aus Anwendung von (1) erhalten wir, dass $L=H$.
Aber es gilt gleichzeitig auch, dass $ \{ 0 \} \neq f(T)H
= f(T) L \subseteq \closure{f(T) g(T) H} = \{ 0 \} $,
was einen Widerspruch darstellt. Somit muss das Spektrum
einelementig sein.
\end{proof}
Zusätzlich brauchen wir noch ein kleines Lemma, um ein
wenig mehr über unitäre Darstellungen zu erfahren.
\begin{lemma}\label{lem:unitaryReprMapsInversesToAdjoints}
Sei $( \rho, V)$ eine Darstellung. Dann sind äquivalent:
\begin{enumerate}
    \item $ \rho$ ist unitär.
    \item $ \rho( g^{-1}) = \rho(g)^*$ für alle $g \in G$.
\end{enumerate}
\end{lemma}
\begin{proof}
    \begin{align*}
        \rho \text{ ist unitär} &\iff \rho(g)^* \text{ 
        unitär }\forall g \in G\\
                                &\iff \rho(g)^{-1} = 
                                \rho(g)^* \forall g \in G\\
                                &\iff \rho( g^{-1}) = 
                                \rho(g)^* \forall g \in G,
    \end{align*}
    wobei im letzten Schritt verwendet wurde, dass $ \rho$
    ein Gruppenhomomorphismus ist.
\end{proof}
Nun können wir das Lemma \hyperref[lem:Schur]{Lemma von Schur} 
beweisen, welches mehr oder weniger ein direktes Korollar 
aus den obigen Lemmata \ref{lem:topologicalSchur} und
\ref{lem:unitaryReprMapsInversesToAdjoints} ist.
\begin{proof}[Beweis (Schur)]
    Nach Lemma \ref{lem:unitaryReprMapsInversesToAdjoints}
    ist $ A\coloneq\{ \rho(g): g \in G \} $ eine Menge, 
    für die $ S \in A \iff S^* \in A$ gilt. Daher können
    wir Lemma \ref{lem:topologicalSchur} anwenden und 
    erhalten die gewünschte Aussage.
\end{proof}
Während das Lemma von Schur im ersten Moment so wirkt,
als könne man damit nur die Homomorphismen einer Darstellung
auf sich selbst klassifizieren, liefert es auch viele
Informationen für die Homomorphismen zwischen beliebigen
irreduziblen Darstellungen.
\begin{korollar}\label{cor:HomSpaceOneDimensional}
Seien $( \rho, V_ \rho), ( \pi, V_ \pi)$ zwei irreduzible,
unitäre Darstellungen. Dann gilt für einen $G$-
Homomorphismus $T: V_ \rho\to V_ \pi$, dass $T=0$ oder dass
$T$ invertierbar mit stetiger Umkehrabbildung ist. In diesem
Fall existiert ein $c>0$, sodass $ cT$ unitär ist. Dies
zeigt insbesondere, dass $\Hom_G(V_ \rho, V_ \pi) = \{ 0 \}$
gilt, außer $ \rho$ und $ \pi$ sind unitär äquivalent.
In diesem Fall ist $\dim(\Hom_G( V_ \rho, V_ \pi)) = 1$-
\end{korollar}
\begin{proof}
Sei $T: V_ \rho \to V_ \pi$ ein $G$-Homomorphismus. 
Betrachte $T^*: V_ \pi \to V_ \rho$. Diese Abbildung ist
ebenfalls ein $G$-Homomorphismus wie folgende Rechnung
zeigt:
\begin{align*}
    \langle \rho(g) T^* x, y \rangle &= \langle T^*x, 
    \rho(g)^* y \rangle\\ 
                                     &= \langle x, T \rho( 
                                     g^{-1}) y \rangle \\
                                     &= \langle x, 
                                 \pi(g^{-1})Ty\rangle\\
                                     &= \langle \pi(g)x, 
                                     Ty \rangle \\
                                     &= \langle T^* \pi(g)x
                                     ,y \rangle, 
\end{align*}
für alle $g \in G, x,y \in H$. Damit ist $T^* T \in 
\Hom_G(V_ \rho, V_ \rho)$. Aus dem \hyperref[lem:Schur]{
Lemma von Schur} folgt, dass $T^* T = \lambda \id$ für
ein $ \lambda \in \C$. Falls $T\neq 0$, dann ist
$T^* T \neq 0$. Durch die positive Semidefinitheit der 
Abbildung, folgt $ \lambda >0$. Betrachte $c = 
\sqrt{ \lambda^{-1}}$. Dann ist $(cT)^*(cT) = \id$. 
Zusätzlich folgt analog, dass $T T^*$ bijektiv ist. Damit
muss $cT$ bijektiv sein und somit auch unitär.\\
Für je zwei $G$-Homomorphismen $S,T: V_ \rho\to V_ \pi$ mit
$T\neq 0$ folgt also, dass $S \circ T^{-1} \in \Hom_G(V_ 
\pi, V_ \pi)$. Somit $S \circ T^{-1} = \lambda \id$ für
ein $ \lambda \in\C$. Somit $S = \lambda T$. Dies zeigt
die Eindimensionalität.
\end{proof}
\section{Kompakte Gruppen}
Das Lemma von Schur liefert uns schon eine gute
Charakterisierung irreduzibler Darstellungen. Jedoch
interessieren uns auch nicht notwendigerweise irreduzible
Darstellungen. Diese können im Allgemeinen etwas
unübersichtlicher werden. Für kompakte Gruppen wird die 
Situation jedoch recht überschaubar und hier wird uns das
erste Mal die Integration bezüglich des Haarmaßes gute 
Dienste erweisen. \\
Zuerst beginnen wir damit, die endlichdimensionalen 
Darstellungen genauer zu studieren. Dies wirkt auf den 
ersten Blick stark einschränkend. Wir werden jedoch sehen,
wie mächtig diese Darstellungen auf kompakten Gruppen 
sind.\\
Bevor wir damit jedoch starten können, müssen wir zumindest
kurz über die Haarmaße auf kompakten Gruppen reden. A
priori ist nämlich nicht klar, ob es einen Unterschied 
zwischen linken und rechten Haarmaßen gibt oder nicht.
Dazu brauchen wir einige Hilfsmittel.
\begin{definition}[modulare Funktion, unimodulare Gruppen]
    Sei $G$ eine lokalkompakte Gruppe und $ \mu$ ein
    Haarmaß auf $G$. Für $x \in G$ definieren wir das Maß
    $ \mu_x$ durch $ \mu_x(A) = \mu(Ax)$, wobei es sich
    ebenfalls um ein Haarmaß handelt. Somit gibt es durch
    die Eindeutigkeit des Haarmaßes eine Zahl $\Delta(x)>0$,
    mit $ \mu_x = \Delta(x) \mu$. Dies definiert eine 
    Abbildung $ \Delta: G\to \R_{>0}$, welche die
    \underline{modulare Funktion}\index{modulare Funktion}
    der Gruppe $G$ genannt wird.\\
    Wenn $ \Delta \equiv 1$, dann heißt $G$ eine
    \underline{unimodulare Gruppe}\index{unimodulare Gruppe}
    .
\end{definition}
\begin{bemerkung}
Für unimodulare Gruppen stimmen linke und rechte Haarmaße
überein.
\end{bemerkung}
Damit wir uns keine Gedanken über das Maß machen müssen,
wollen wir zeigen, dass kompakte Gruppen unimodular sind.
\begin{satz}\label{thm:propsModularFunction}
Sei $G$ eine lokalkompakte Gruppe mit modularer Funktion
$ \Delta: G\to \R^\times_{>0}$. Dann gelten folgende 
Eigenschaften:
\begin{enumerate}
    \item Für $ y \in G$ und $ f \in \LL^1(G)$ ist
        $R_y f \in \LL^1(G)$ und es gilt
        $$ \int_{G} R_y f(x) \,\diff x = \int_{G} f(xy) 
        \,\diff x = \Delta( y^{-1}) \int_{G} f(x) 
        \,\diff x,$$
        wobei $R_yf: G\to \R, x \mapsto f(xy)$.
    \item $ \Delta$ ist ein stetiger Gruppenhomomorphismus.
    \item Kompakte Gruppen sind unimodular.
\end{enumerate}
\end{satz}
\begin{proof}
    \begin{enumerate}
        \item Für charakteristische Funktionen 
            $\mathds{1}_A$ gilt:
            \begin{align*}
                \int_{_G} \mathds{1}_A(xy) \,\diff x &=
                \int_{A} xy \,\diff x \\
                &= \int_{Ay^{-1}} x \,\diff  x\\
                &= \mu(A y^{-1})\\
                &= \Delta( y^{-1}) \mu(A)\\
                &= \Delta( y^{-1}) \int_{G} \mathds{1}_A(x)
                \,\diff x.
            \end{align*}
            Für beliebige integrierbare Funktionen folgt die
            Aussage mit maßtheoretischer Induktion.
        \item Für $x,y \in G$ und eine messbare Menge
            $A \subseteq G$ gilt
            \begin{align*}
                \Delta(xy) \mu(A) &= \mu_{xy}(A) \\
                                  &= \mu(Axy)\\
                                  &= \mu_y(Ax)\\
                                  &= \Delta(y) \mu(Ax)\\
                                  &= \Delta(y) \Delta(x) 
                                  \mu(A).
            \end{align*}
            Durch Wahl einer beliebigen Menge $A$ mit
            $ 0 < \mu(A) < \infty$ gilt dann
            $$ \Delta(xy) = \Delta(x) \Delta(y).$$
            Zur Stetigkeit in $y$ wähle eine Funktion
            $f \in C_c(G)$ mit $ c = \int_{G} f(x)  \,\diff
            x > \neq 0$. Dann folgt durch den ersten Teil
            der Aussage 
            $$ \Delta(y) = \frac{1}{c} \int_{G} f(x y^{-1}) 
            \,\diff x  = \frac{1}{c} \int_{G} R_{ y^{-1}} 
            f(x) \,\diff  x.$$
            Dieser Ausdruck ist stetig in $y$, weswegen auch
            $ \Delta$ stetig in $y$ sein muss.
        \item Sei nun $G$ kompakt. Da $ \Delta$ ein
            stetiger Gruppenhomomorphismus ist, muss
            $ \Delta(G)$ eine kompakte Untergruppe von
            $ \R_{>0}^\times$ sein. Die einzige kompakte
            Untergruppe dieser Gruppe ist jedoch $ \{ 1 \}$.
    \end{enumerate} 
\end{proof}
Dass wir nicht zwischen linken und rechten Haarmaßen
unterscheiden müssen, erleichtert die Situation erheblich,
weswegen wir uns nun den endlichdimensionalen Darstellungen
widmen können. Sei also im Folgenden stets $K$ eine kompakte
(also insbesondere lokalkompakte) topologische Gruppe sowie
$ (\rho, V)$ eine endlichdimensionale Darstellung, sofern
nicht anders gefordert.
\begin{lemma}
\label{lem:finiteDimensionalRepresentationsAlwaysUnitary}
Auf $V$ gibt es stets ein Skalarprodukt, sodass $ \rho$
eine unitäre Darstellung wird. Falls $ \rho$ irreduzibel
ist, dann ist dieses Skalarprodukt eindeutig bis auf
eine positive Konstante bestimmt.
\end{lemma}
\begin{proof}
Wir wählen ein beliebiges Skalarprodukt $( \cdot, \cdot): 
V\times V\to \C$. Dann definieren wir ein neues 
Skalarprodukt durch den Ausdruck
$$ \langle v, w \rangle \coloneq \int_{K} (
 \rho(k)v, \rho(k)w)\,\diff k$$
bezüglich des Haarmaßes $ \mu$, für das $ \mu(K) = 1$ gilt.
Wir zeigen nun, dass dies tatsächlich ein Skalarprodukt
definiert:
\begin{enumerate}
    \item Die Linearität folgt direkt aus der Linearität
        von $ (\cdot, \cdot)$ und des Integrals.
    \item Dass das Skalarprodukt hermitesch ist, folgt
        unmittelbar aus der Hermitizität des ursprünglichen
        Skalarproduktes und dem Fakt, dass Konjugieren und
        Integrieren kommutieren.
    \item Es bleibt nur noch zu zeigen, dass die so
        definierte Abbildung positiv definit ist. Dabei ist
        die Positivität ebenfalls klar. Sei also $ v \in V$,
        sodass $ \langle v,v \rangle =0$. Dann gilt
        $$ 0 = \langle v,v \rangle = \int_{K} ( \tau(k)v, 
        \tau(k)v)  \,\diff k.$$
        Die Abbbildung $ k \mapsto (\tau(k)v, 
        \tau(k)v) $ ist stetig und somit können
        wir obiges Korollar aus der Maßtheorie %TODO 
        verwenden und erhalten, dass $ k \mapsto ( \tau(k)v
        , \tau(k)v)$ gleich der Nullabbildung ist. 
        Insbesondere gilt also für $k = 1_G$, dass
        $ \tau(1_G) = \id_V$. Somit folgt also ebenfalls
        $ (v, v) = 0$. Da $(\cdot, \cdot)$ ein Skalarprodukt
        ist, folgt $ v = 0$.
\end{enumerate}
Damit ist $ \langle \cdot, \cdot \rangle $ ein Skalarprodukt
und wir müssen nur noch zeigen, dass $ \rho$ unitär 
bezüglich diesen Skalarprodukts ist. Dazu sehen wir, dass
\begin{align*}
    \langle \tau(x)v, \tau(x)w \rangle 
    &= \int_{K}(\tau(k)\tau(x)v, \tau(k)\tau(x)w)\,\diff k\\
    &= \int_{K}( \tau(kx)v, \tau(kx)w) \,\diff  k\\
    &= \int_{K} ( \tau(k)v, \tau(k)w) \,\diff k\\
    &= \langle v,w \rangle, 
\end{align*}
da die Gruppe $K$ unimodular ist und es sich mit $f:
K\to \C, k\mapsto ( \tau(k)v, \tau(k)w)$ bei
$R_x f : K\to \C$ genau um $k \mapsto ( \tau(kx)v, 
\tau(kx)w)$ handelt und wir somit 
Satz \ref{thm:propsModularFunction} anwenden können. Was
noch zu zeigen ist, ist die Eindeutigkeit bis auf 
positive Konstanten im Fall irreduzibler Darstellungen. 
Dazu seien $( \rho_1, V_1), (\rho_2, V_2)$ Darstellungen,
die sich nur im unitären Skalarprodukt auf $V$ 
unterscheiden. Bezeichne diese Skalarprodukte mit 
$ \langle \cdot, \cdot \rangle_1$ und 
$ \langle \cdot,\cdot \rangle_2$. Betrachte nun die
lineare Abbildung $\id:V_1\to V_2, v\mapsto v$. Da
$V_1$ und $V_2$ endlichdimensional ist, ist $\id$ 
stetig und ein $G$-Homomorphismus zwischen $ \rho_1$
und $\rho_2$. Nun nutzen wir Korollar
\ref{cor:HomSpaceOneDimensional}, um zu folgern, dass
es ein $c>0$ gibt, sodass $c\cdot \id$ unitär zwischen
den beiden Darstellungen ist. Insbesondere folgt dann
also, dass 
\begin{align*}
	c^2 \langle v_1, v_2 \rangle_2 
	&= \langle c v_1, c v_2 \rangle_2 \\
	&= \langle v_1, v_2 \rangle_1
\end{align*}
Damit sind die Skalarprodukte bis auf positive Konstante
gleich.
\end{proof}
Zusätzlich lassen sich die endlichdimensionalen 
Darstellungen gut in "Atome" -- die irreduziblen 
Darstellungen -- zerlegen.
\begin{lemma}\label{lem:FiniteDimReprSumOfIrredRepr}
Sei $( \rho, V)$ eine endlichdimensionale Darstellung 
von $K$. Dann gibt es irreduzible Darstellungen 
$(\rho_i, V_i)_{i \in I}$, sodass $$ \rho = 
\bigoplus_ { i \in I } \rho_i$$.
\end{lemma}
\begin{proof}
Sei $K$ eine kompakte, topologische Gruppe und $(\rho, V)$
eine Darstellung von $K$. Da $V$ endlichdimensional ist,
können wir die Aussage mit Induktion über $\dim(V)$
führen.\\
\underline{I.A. ($\dim(V)=1$):} Dieser Fall ist klar, da die
einzigen Unterräume von $V$ $ \{ 0 \} $ und $V$ sind.
Somit muss jede eindimensionale Darstellung schon
direkt irreduzibel sein.\\
\underline{I.S.:} Sei $ V$ ein Vektorraum der Dimension
$ n+1$ für ein beliebiges $ n \in \N_0$. Angenommen, wir 
haben die Aussage schon für alle Vektorräume der 
Dimension kleiner gleich $n$ gezeigt. Zusätzlich können
wir mit Lemma 
\ref{lem:finiteDimensionalRepresentationsAlwaysUnitary}
annehmen, dass $ \rho$ eine unitäre Darstellung ist. 
Falls $\rho$ bereits irreduzbiel ist, sind wir fertig.
Ansonsten gibt es einen $\rho(K)$-invarianten 
Untervektorraum $ \{ 0 \} \subsetneq  U \subsetneq V$. 
Betrachte dessen orthogonales Komplement $U^\perp$. 
Auch dieser ist abgeschlossen unter $\rho(K)$, denn 
für $ g \in K, u \in U, v \in U^\perp$ gilt:
\begin{align*}
	\langle u, \rho(g)v \rangle 
	&= \langle \rho(g)^*u, v \rangle \\
	&= \langle \rho(g^{-1})u, v \rangle\\
	&= 0,
\end{align*}
wobei wir im vorletzten Schritt verwendet haben,
dass $ \rho(g)$ unitär ist, und im letzten
Schritt, dass $U$ $\rho(K)$-invariant ist.
Nun können wir die Induktionsvoraussetzung auf
$U$ und $U^\perp$ anwenden und erhalten durch
$ V = U \oplus U^\perp$ die gewünschte Aussage.
\end{proof}
Wir beenden diesen Abschnitt mit zwei Definitionen
und einem Korollar:
\begin{definition}[unitäres Dual, endliches unitäres Dual]
Sei $G$ eine lokalkompakte Gruppe. Dann bezeichnen
wir mit $\UDual{G}$ die Menge aller Äquivalenzklassen
irreduzibler Darstellungen auf $G$ bezüglich
unitärer Äquivalenz. Wir nennen $\UDual{G}$
\underline{unitäres Dual}\index{unitäres Dual}
zu $G$.\\
Zusätzlich bezeichnen wir für eine kompakte Gruppe $K$
die Äquivalenzklassen der endlichdimensionalen,
irreduziblen Darstellungen mit $\UDualFin{K}$ und
nennen dies das \underline{endliche, unitäre Dual}
\index{endliche, unitäre Dual} von $K$.
Darstellungen
\end{definition}
\begin{bemerkung}
A priori gibt es ein kleines mengentheoretisches Problem
bei der Definition dieser beiden Mengen, da nicht trivial
ist, dass es sich wirklich um Mengen handelt. Um dies zu
beheben, müssten wir ein wenig in die Theorie der 
Kardinalzahlen eintauchen. Dies wollen wir hier nicht
weiter vertiefen.
\end{bemerkung}
Ein wesentliches Ziel des Vortrags ist es zu zeigen,
dass $\UDual{K} = \UDualFin{K}$ gilt.
\begin{korollar}
	Es gilt stets $\UDualFin{K}\subseteq \UDual{K}$.
\end{korollar}
\begin{proof}
Zu zeigen ist bloß, dass alle endlichdimensionalen
Darstellungen auch unitär sind. Dies haben wir 
bereits in Lemma 
\ref{lem:finiteDimensionalRepresentationsAlwaysUnitary}
gezeigt.
\end{proof}
\chapter{Der Satz von Peter-Weyl}
\section{Orthogonalitätseigenschaften}
Zuerst wollen wir eine gewisse Verallgemeinerung von
Matrixelementen auf unitäre Darstellungen definieren. 
\begin{definition}[Matrixkoeffizienten]
Sei $K$ eine kompakte, topologische Gruppe und $( \tau, V)$
eine unitäre Darstellung von $K$. Die 
\underline{Matrixkoeffizienten}\index{Matrixkoeffizienten}
von $ \tau$ sind Abbildungen der Form $ k\mapsto 
\langle \tau(k)v, w \rangle $ für beliebige $v,w \in V$.
\end{definition}
Für die Matrixkoeffizienten können wir einige spannende
Eigenschaften feststellen:
\begin{lemma}\label{lem:propsMatrixCoeffs}
Sei $K$ eine kompakte, topologische Gruppe und $( \tau, V)$
eine endlichdimensionale Darstellung. Dann gilt
\begin{enumerate}
    \item Die Matrixkoeffizienten von $ \tau$ sind stetig.
    \item Die Matrixkoeffizienten von $ \tau$ sind 
        $\LL^2(K)$-Funktionen.
\end{enumerate}
\end{lemma}
\begin{proof}
Da $K$ kompakt ist, folgt die zweite Aussage direkt aus 
der ersten. Zu zeigen ist also nur die Stetigkeit.\\
Da $ \tau$ eine Darstellung ist, ist die Abbildung
$$ K\times V\to V, (k, v)\mapsto \tau(k)v$$
stetig. Insbesondere ist für festes $ v \in V$ also auch 
die Abbildung
$$ K\to V, k \mapsto \tau(k)v$$
stetig. Da Skalarprodukte auf Hilberträumen stets stetig
sind, folgt, dass die Abbildung 
$$ \varphi: K\to \C, k \mapsto \langle \tau(k)v, w 
\rangle $$
stetig, da eine Komposition stetiger Abbildungen ist.
\end{proof}
Im Folgenden wollen wir zeigen, dass es sich bei den
Matrixkoeffizienten um eine Orthonormalbasis des Raums
$\LL^2(K)$ handelt. \\
Dazu wählen wir uns zuerst einen Repräsentanten 
$( \tau, V_ { \tau })$ jeder Äquivalenzklasse aus 
$\UDualFin{K}$ sowie eine Orthonormalbasis $e_1, \dots, e_n$
jedes $ V_ \tau$.\\
Dann führen wir folgende Notation ein, um die nachfolgenden
Sätze ein wenig angenehmer aufschreiben zu können:
$$ \tau_{ij}: K \to \C, k\mapsto \langle \tau(k)e_i, e_j 
\rangle.$$
Also meinen wir damit den Matrikoeffizienten von $k$
an der Stelle $(i, j)$. 
\begin{satz}\label{thm:orthoPropsMatrixCoeffs}
Mit obiger Notation gelten folgende Aussagen:
\begin{enumerate}
    \item Für $ \tau \neq \rho$ gilt für alle $1 
    \leq i,j \leq \dim(V_ \tau), 1 \leq r,s \leq 
    \dim(V_ \rho)$:
        $$\langle \tau_{ij}, \rho_{rs} \rangle_{\LL^2} =0.$$
    \item Für $ \tau$ gilt
        $$ \langle \tau_{ij}, \tau_{rs} \rangle = 
        \frac{1}{\dim(V_ \tau)}\delta_{ir} \delta_{js}.$$
\end{enumerate}
\end{satz}
Bevor wir den Satz beweisen, ziehen wir ein unmittelbares
Korollar.
\begin{korollar}\label{cor:matrixCoeffsOrthoSys}
    $ ( \dim(V_ \tau) \tau_{ij})_{ \tau, i, j}$ bildet ein 
    Orthonormalsystem in $ \LL^2(K)$.
\end{korollar}
Nun beweisen wir den obigen Satz 
\ref{thm:orthoPropsMatrixCoeffs}:
\begin{proof}
\begin{enumerate}
    \item Seien $ \tau \neq \rho $ Repräsentanten aus 
        $\UDualFin{K}$.\\
        Dann betrachte eine lineare Abbildung $ T: V_ \tau 
        \to V_ \rho$ und definiere dafür die Abbildung
        $S_T: V_ \tau\to V_ \rho$ durch 
        $$S_T = \int_{K} \rho( k^{-1}) T \tau(k) \,
        \diff k.$$
        Wir bezeichnen $S_T$ ab jetzt nur noch mit $S$.
        Dann gilt die folgende Gleichheit: 
        $S \tau(k) = \gamma(k) S$, also ist $S$ ein 
        Vertauscher der Darstellungen $ \tau$ und $ \rho$.
        Dazu sehen wir folgende Rechnung: 
	\begin{align*}
		\rho(k) S 
		&= \rho(k)\left(\int_{K}\rho( m^{-1}) 
			T \tau(m)\,\diff m \right)\\
		&= \int_{K} \rho(k)\rho( m^{-1}) T  
		        \tau(m)\,\diff m  \\
		&= \int_{K} \rho(k m^{-1}) T \tau(m) 
		\,\diff m\\
		&= \int_{K} \rho( (m k^{-1})^{-1}) 
		T \tau(m k k^{-1})\,\diff m \\
		&= \int_{K} \rho(m k^{-1})^* T 
		\tau(m k k^{-1}) \,\diff m  \\
	\end{align*}
	Nun können wir den Satz \ref{thm:propsModularFunction}
	verwenden und folgern, dass
	\begin{align*}
		\rho(k)S 
		&= \int_{K} \rho(m)^* T \tau(mk) 
		\,\diff m\\
	        &= \int_{K} \rho( m^{-1}) T \tau(m) \tau(k) 
		\,\diff m \\
		&= \left(\int_{K} \rho( m^{-1}) T \tau(m) 
		\,\diff m \right)( \tau(k))\\
		&= S \tau(k)
	\end{align*}
	Darauf können wir nun das Lemma von Schur bzw. das 
	Korollar \ref{cor:HomSpaceOneDimensional} verwenden.
	Denn da $ \tau \neq \rho$ gilt, sind die beiden
	Darstellungen nicht unitär äquivalent. Oben haben
	wir jedoch nachgerechnet, dass die Abbildung $S$
	ein Vertauscher zwischen den beiden Darstellungen
	ist. Somit muss $S=0$ gelten.\\
	Nun wollen wir die Eigenschaften solcher $S_T$ 
	ausnutzen. Dafür wählen wir zuerst zwei 
	Orthonormalbasen $(e_j)$ und $(f_s)$ von $V_ \tau$
	bzw. $ V_ \rho$. und betrachten die lineare
	Abbildung $ T_ { js } $, welche definiert wird
	durch $T_{ js}(v) = \langle v, e_j \rangle f_s$.
	Für diese Abbildung wählen wir jetzt $S_ { js } 
	\coloneq S_ { T_ { js }   }$. Dann erhalten
	wir unmittelbar unter Ausnutzung der obigen 
	Rechnung
	\begin{align*}
		0 
		&= \langle S_{js}e_i, f_r \rangle \\
		&= \int_{K} \langle \rho( k^{-1})
		T_{js} \tau(k)e_i, f_r \rangle  \,\diff k\\
		&= \int_{K} \langle \rho( k^{-1})
		\langle \tau(k)e_i, e_j \rangle f_s, f_r 
		\rangle \,\diff k \\
		&= \int_{K} \langle \tau(k)e_i, e_j \rangle  
		\langle \rho( k^{-1})f_s, f_r \rangle 
		\,\diff k\\
		&= \int_{K} \langle \tau(k)e_i, e_j \rangle
		\langle f_s, \rho(k) f_r \rangle \,\diff k\\
		&= \int_{K} \langle \tau(k)e_i, e_j \rangle 
		\overline{\langle \rho(k)f_r, f_s  \rangle} 
		\,\diff k\\
		&= \int_{K} \tau_{ij}(k) 
		\overline{\rho(k)_{rs}}  \,\diff k \\
		&= \langle \tau_{ij}, \rho_{rs} 
		\rangle_{\LL^2}.
	\end{align*}
	Dies zeigt den ersten Teil der Aussage.
    \item Nun betrachten wir den Fall, dass wir den
	    gleichen Repräsentanten der Darstellungen
	    verwenden, also $ \langle \tau_{ij}, \tau_{rs} 
	    \rangle_{\LL^2}$. Obige Rechnung liefert 
	    erneut $$ \langle S_{js}e_i, e_r \rangle 
	    = \langle \tau_{ij}, \tau_{rs}\rangle_{\LL^2}.$$
	    Nun ist jedoch $S \in \Hom_{G}(V_\tau, V_\tau)$.
	    Damit folgt aus dem \hyperref[lem:Schur]
	    {Lemma von Schur}, dass $S_{js} = \lambda 
	    \id_{ V_ \tau}$. Somit muss für $i \neq r$
	    folgen, dass $\langle S_{js} e_i, 
	    e_r\rangle=0$.\\
	    Wir wollen nun noch zeigen, dass auch für
	    $j \neq s$ die Gleichheit mit $0$ folgt. Seien
	    also $j \neq s$. Wir zeigen dann sogar,
	    dass $S_{js} = 0$. Dazu betrachten wir ein
	    typisches Argument in der Theorie irreduzibler 
	    Darstellungen. Es gilt:
	    \begin{align*}
		   \tr( S_{js}) 
           &= \tr\left( \int_{K} \tau( k^{-1}) T_{js} 
           \tau(k) \,\diff  k\right)\\
           &= \int_{K}  \tr( \tau(k)^{-1} T_{js} \tau(k))
           \,\diff  k\\
           &= \int_{K} \tr( \tau(k) \tau(k)^{-1} T_{js})
           \,\diff  k\\
           &= \int_{K} \tr( T_{js}) \,\diff k
	    \end{align*}
	    Wenn wir nun jedoch die Darstellungsmatrix von
        $T_{js}$ bezüglich der gewählten Orthonormalbasis
        $(e_i)$ betrachten, dann fällt auf, dass $j\neq s$
        impliziert, dass auf der Diagonale nur $0$ steht.
        Somit ist $ \tr( T_{js}) = 0$ und somit auch
        $ \tr(S_{js}) = 0$. Gleichzeitig folgt aus 
        $S_{js} = \lambda \id$ auch, dass 
        $ \tr(S_{js}) = \lambda^{\dim(V_ \tau)}$. Somit
        muss $ \lambda = 0 $ gelten und damit ist 
        $S_{js} = 0$. Dies zeigt wieder, dass 
        $ \langle \tau_{ij}, \tau_{rs} \rangle_{\LL^2}=0$
        ist, wenn $i \neq r$ oder $ j \neq s$.\\
        Was noch zu zeigen bleibt, ist der Fall $i = r$
        und $j=s$. Wir erhalten $S_{jj} = \lambda_j \id$
        für alle $j$. Somit folgt aus obiger Rechnung, dass
        $$ \lambda_j = \langle \lambda_j e_i, e_i \rangle  
        = \langle S_{jj}e_i, e_i \rangle = \langle \tau
        _{ij}, \tau_{ij} \rangle_{\LL^2}$$
        und somit $ \langle \tau_{ij}, \tau_{ij} 
        \rangle_{\LL^2} $ nicht von $i$ abhängt. Ziel ist
        es jetzt, zu zeigen, dass alle $ \lambda_j$ gleich
        sind. Da$ \tau_{ij}$ eine unitäre Darstellung ist, 
        folgt $ \tau_{ij}(k)=\overline{\tau_{ji}( k^{-1})}$.
        Somit gilt:
        \begin{align*}
            \lambda_j 
            &= \langle \tau_{ij},\tau_{ij}\rangle_{\LL^2}\\
            &= \int_{K} \tau_{ij}(k)\overline{\tau_{ij}(k)}
            \,\diff k \\
            &= \int_{K} \overline{\tau_{ji}(k^{-1})} 
            \tau_{ji}( k^{-1})\,\diff k \\
            &= \int_{K} \overline{ \tau_{ji}( k)}
            \tau_{ji}( k)\,\diff k\\ %TODO: den Pushforward
            % der Inversionsabbildung aufschreiben und hier
            % sauber referenzieren
            &= \langle \tau_{ji},\tau_{ji} \rangle_{\LL^2}\\
            &= \lambda_i
        \end{align*}
        wobei im vorletzten Schritt Satz ?? %TODO: hier referenz einbauen
        verwendet wurde. Somit muss es sich stets um den
        gleichen Wert bei allen $ \lambda_j$ handeln.
        Diesen nennen wir $ \lambda$. Wir müssen jetzt 
        nur noch zeigen, dass $ \lambda = 
        \frac{1}{\dim(V_ \tau)}$ gilt. Dazu schreiben
        wir $ n\coloneq \dim(V_ \tau)$. Durch Prüfen auf
        unserer gewählten Basis $(e_i)$ sehen wir schnell
        ein, dass $ \sum_{i} T_{ii} = \id $. Somit folgt
        \begin{align*}
            (n \lambda)\id 
            &= \sum_{i} S_{ii}\\
            &= \sum_{i} \int_{K} \tau( k^{-1}) T_{ii} 
             \tau(k)\,\diff k\\
            &= \int_{K} \tau( k^{-1}) \sum_{i}T_{ii} \tau(k)
            \,\diff  k\\
            &= \int_{K} \tau( k^{-1}) \id\tau(k) \,\diff k\\
            &= \id
        \end{align*}
        Somit muss $ \lambda = \frac{1}{n}$ folgen, was die
        zu zeigende Aussage beweist.
\end{enumerate} 
\end{proof}
Damit wissen wir schon einmal, dass die Matrixkoeffizienten
ein Orthonormalsystem bilden. Der Frage, ob sie auch eine
Orthonormalbasis bilden, werden wir im nächsten Abschnitt 
nachgehen, wenn wir den Satz von Peter-Weyl beweisen werden.
\section{Vollständigkeitseigenschaften}
In diesem Abschnitt werden wir mehrere Sätze beweisen,
die unter dem Satz von Peter-Weyl bekannt sind. Dazu benötigen
wir jedoch noch ein wenig Vorarbeit:
\begin{definition}[Dirac-Funktionen, Dirac-Netz]
Auf einer lokalkompakten, topologischen Gruppe $G$ heißt eine
Funktion $ \phi \in C_c(G)$ eine \underline{Dirac-Funktion}
\index{Dirac-Funktion}, falls
\begin{itemize}
	\item $ \phi \geq 0$,
	\item $ \int_{G} \phi(x) \,\diff x =1  $,
	\item $ \phi( x^{-1}) = \phi(x)$.
\end{itemize}
Sei $\mathcal U$ die Menge aller offenen Umgebungen der $1_G$.
Dann heißt eine Familie von Dirac-Funktionen $( \phi_U)_ 
{ U \in \mathcal U  } $ ein \underline{Dirac-Netz}
\index{Dirac-Netz}, falls $ \supp{ \phi_U} \subseteq U$
für alle $U \in \mathcal{U}$ gilt.
\end{definition}
% TODO: Faltung und Faltung glättet L²-Funktionen zu C-
% Funktionen
\begin{satz}[Peter-Weyl I]
    Sei $K$ eine kompakte, topologische Gruppe sowie 
    $ \tau$ Repräsentanten der Äquivalenzklassen aus 
    $\UDualFin{K}$ mit dem zugehörigen Orthonormalsystem
    $( \tau_{ij})_{ \tau, i, j}$ aus 
    \ref{thm:orthoPropsMatrixCoeffs}. Dann gilt
    \begin{enumerate}
        \item dieses Orthonormalsystem ist sogar eine
            Orthonormalbasis.
        \item Die linksreguläre Darstellung $(L, \LL^2(K))$
            aus Beispiel \ref{ex:leftRegularRepr}
            zerfällt in eine direkte Summe 
            endlichdimensionaler, irreduzibler 
            Darstellungen.
    \end{enumerate} 
\end{satz}
\begin{proof}
\begin{enumerate}
	\item Sei $ \tau \in \UDualFin{K}$ und $ M_ \tau$ der
	Unterraum, der durch die Matrixkoeffizienten $ M_ 
	\tau$ aufgespannt wird. Zuerst sehen wir schnell ein,
	dass für einen Matrixkoeffizienten $ h(k) = 
	\langle \tau(k)v, w  \rangle $ gilt
	\begin{align*}
		h^*(k) &= \bar{h( k^{-1})} = \langle \tau(k)
		w,v \rangle \in M_ \tau\\
		L_{k_0}h(k) &= h( k_0^{-1}k) = \langle 
		\tau(k)v, \tau(k_0)w \rangle \in M_ \tau\\
		R_{k_0}h(k) &= h( k k_0) = \langle \tau(k) 
		\tau(k_0)v, w \rangle in M_ \tau.
	\end{align*}
	Somit ist $ M_ \tau$ abgeschlossen unter Links- und
	Rechtstranslationen sowie Konjugation. Nun 
	bezeichnen wir mit $M$ den $\LL^2(K)$-Abschluss 
	aller $M_ \tau$. Wir wollen zeigen, dass $M = 
	\LL^2(K)$ oder äquivalent, dass $ M^\perp = \{ 0 \}$.
	Angenommen, $M^\perp \neq \{ 0 \}$. 
	\begin{claim}
		$\orthoComp{M}$ 
		enthält eine stetige Funktion ungleich	der
		Nullfunktion.
	\end{claim}
	Sei $ H \neq 0$ in $M^\perp$ und wähle ein Dirac-Netz
	$ (\phi_U)_U$. Dann konvergiert $ \phi_U \ast H$
	in der $\LL^2$-Norm gegen $H$. Da $\orthoComp{M}$ 
	abgeschlossen unter Translationen ist, und $ \phi_U
	\in C_c(K)$ für alle $U \in \mathcal{U}$ folgt 
	direkt, dass $ \phi_U \ast H \in \orthoComp{M}$.
	Da die Faltung Funktionen glättet und es ein $U \in
	\mathcal{U}$ geben muss mit $ \phi_U \ast H \neq 0$
	(sonst wäre $H$ durch die $\LL^2$-Konvergenz bereits
	die Nullfunktion), folgt die Behauptung.\\
	Wir wählen also eine stetige Funktion $F_1 \in 
	\orthoComp{M} $ mit $ F_1 \neq 0$. Durch Translation
	und Reskalierung können wir annehmen, dass $F_1(e)
	>0$.
\end{enumerate}

\end{proof}

\section{Anwendung auf Lie-Gruppen}
Für die Situation von Lie-Gruppen können wir aus dem 
Satz von Peter und Weyl die Existenz treuer Darstellungen
von Lie-Gruppen folgern.
\begin{definition}[treue Darstellung]
    Eine Darstellung auf einer topologischen Gruppe $G$
    $ \rho: G\to \GLS(V)$ und einem Vektorraum $V$
    heißt \underline{treu}\index{treue Darstellung},
    falls injektiv ist.
\end{definition}
\begin{bemerkung}
Eine Darstellung $ \rho$ ist also genau dann treu, wenn
$\ker( \rho) = \{ e_G \} $ ist. Das bedeutet also, dass
nur das triviale Element der "Nullwirkung" entspricht.
\end{bemerkung}
Dann gilt die folgende Eigenschaft:
\begin{satz}\label{thm:compLieGrpsHaveFaithfulRepr}
    Sei $K$ eine kompakte Lie-Gruppe. Dann besitzt $K$
    eine treue Darstellung.
\end{satz}


\printindex
\printbibliography
\end{document}
