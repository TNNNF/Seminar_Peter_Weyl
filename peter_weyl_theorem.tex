\documentclass{report}

\usepackage{style}

\begin{document}
\tableofcontents
\chapter{Einführung}
\section{Motivation}
Das zentrale Ergebnis des letzten Vortrags war die Existenz
und Eindeutigkeit des Haarmaßes, welche sich im folgenden
Satz zusammenfassen lässt.
\begin{satz*}[Existenz und Eindeutigkeit des Haarmaßes]
    Sei $G$ eine lokalkompakte Gruppe. Dann gibt es eine 
    linksinvariantes Radonmaß $ \mu\neq 0$ auf 
    $\mathcal{B}(X)$. Dieses ist eindeutig bis auf 
    positive Vielfache.
\end{satz*}
Dieses Maß wird als das Haarmaß bezeichnet. Somit erhalten
wir auf jeder Gruppe einen sinnvollen Begriff des Integrals.
Dies erlaubt uns, wirklich sinnvoll über Räume wie 
$\LL^2(G)$ zu reden.\\
Im Spezialfall erhalten wir dann natürlich Ergebnisse für
die Integrationstheorie auf dem $\R^d$ oder anderen 
spannenden Gruppen wie $\R^\times$. Jedoch sind nicht alle
Gruppen so gut strukturiert wie diese Gruppen. Diese sind
nämlich abelsch und lokalkompakt. Die Integrationstheorie
auf solchen Gruppen ist zentraler Gegenstand der 
\emph{harmonischen Analysis}. Der nicht abelsche Fall ist
jedoch nicht so gut strukturiert und deutlich 
unübersichtlicher. Für den kompakten Fall erhalten wir 
jedoch den \emph{Satz von Peter-Weyl}, welcher viele 
Informationen über die Struktur von $\LL^2(G)$ beinhaltet.
Diesen zu verstehen und zu beweisen ist Gegenstand des 
vorliegenden Seminarvortrags.\\
Zunächst brauchen wir jedoch noch Grundlagen der
Darstellungstheorie topologischer Gruppen.
\section{Notation}
\begin{itemize}
    \item Für einen Banachraum $V$ bezeichnen wir mit 
        $\GLS{V}$ die Menge der stetigen, linearen, 
        invertierbaren Abbildungen.
    \item Für zwei Banachräume $V, W$ bezeichnen wir mit
        $ \mathcal L(V, W)$ die Menge der beschränkten,
        stetigen linearen Abbildungen $T:V\to W$.
\end{itemize}

\chapter{Grundlagen der Darstellungstheorie}
Ziel der Darstellungstheorie ist es gewissermaßen, das 
Problem der schwierigen Strukturierung von Gruppen zu
lösen, indem man die Elemente in Matrizen übersetzt.
Für diese kennen wir nämlich aus der linearen Algebra viele
hilfreiche Sätze, mit denen wir sehr häufig mehr über die
Struktur der Gruppe verstehen können.
\section{Grundlegende Begriffe}
Der zentrale Begriff in der Darstellungstheorie 
topologischer Gruppen ist der einer Darstellung.
\begin{definition}[Darstellung]
Sei $V$ ein Banachraum und $G$ eine topologische Gruppe. 
Eine \underline{Banachraum-Darstellung}
\index{Banachraum-Darstellung}\footnote{Häufig werden wir 
nur Darstellung sagen und damit eine Banachraum-Darstellung 
meinen.} von $G$ auf $V$ ist ein Gruppenhomomorphismus 
$ \pi: G\to \GLS{V}$, sodass die Abbildung $$G\times V\to V,
(g,v)\mapsto (g,v)\mapsto \pi(g)v$$ stetig ist.
\end{definition}
\begin{beispiel}
Für $G=\SL_2(\R)$ erhalten wir folgende 
Standarddarstellung auf $V= \C^2$ gegeben durch $\pi: 
\SL(2,\R)\to \GL(\C^2), A\mapsto A$ und mit 
Matrixmultiplikation als zugehöriger Wirkung.
\end{beispiel}
Bereits aus der Algebra 1 wissen wir um die Wichtigkeit 
bestimmter "elementarer" Strukturen, z.B. den zyklischen
Gruppen bei der Klassifikation endlicher abelscher Gruppen.
Auch in der Darstellungstheorie gibt es einen ähnlichen
Begriff:
\begin{definition}[Unterdarstellungen, irreduzible 
    Darstellungen]
Sei $( \rho, V_ \rho)$ eine Darstellung einer topologischen
    Gruppe $G$. \begin{enumerate}
    \item $( \pi, V_ \pi)$ heißt \underline{
        Unterdarstellung}\index{Unterdarstellung} von 
        $ ( \rho, V_ \rho)$, falls $ V_ \pi$ ein 
        abgeschlossener Untervektorraum von $ V_ \rho$
        ist und $ \rho$ eingeschränkt auf $ V_ \pi$
        gleich $ \pi$ ist.
    \item $( \rho, V_ \rho)$ heißt \underline{irreduzibel}
        \index{irreduzibel}, falls es keine nicht-trivialen
        Unterdarstellungen gibt, also falls für jede 
        Unterdarstellung $( \pi, V_ \pi)$ gilt, dass
        $V_ \pi = \{0\}$ oder $ V_ \pi = V_ \rho$.
\end{enumerate}
\end{definition}
\begin{bemerkung}
Jeder abgeschlossene Untervektorraum $U \subseteq V_ \rho$, 
für den $ \rho(G) U \subseteq U$ gilt, liefert eine 
Unterdarstellung durch Einschränken von $ \rho$ auf $ U$.
\end{bemerkung}
Für den Fall, dass die Banachräume $V$ für unsere 
Darstellungen sogar Hilberträume sind, gibt es noch
eine weitere Art von Darstellungen:
\begin{definition}[Unitäre Darstellungen]
Sei $V$ ein Hilbertraum und $G$ eine topologische Gruppe.
Eine Darstellung $ \rho$ auf $ V$ heißt \underline{unitäre 
Darstellung}\index{unitäre Darstellung}, falls 
$ \rho(g)$ unitär ist für alle $g \in G$, also
$ \langle \rho(g)v, \rho(g)w \rangle = \langle v, w\rangle$
für alle $ g \in G$ und alle $ v,w \in V$.
\end{definition}
Besonders relevant werden auch Beziehungen zwischen
Darstellungen, die wie üblich über Abbildungen gegeben sind.
\begin{definition}[ $G$-Homomorphismen, Äquivalenz von 
    Darstellungen]
    Seien $( \rho, V_ \rho), ( \pi, V_ \pi)$ zwei
    Darstellungen von $G$.
    \begin{enumerate}
        \item Ein stetiger, linearer Operator $T: V_ \rho 
            \to V_ \pi$ heißt \underline{$G$-Homomorphismus}
            \index{$G$-Homomorphismus}, falls
            $$ T \circ \rho(g) = \pi(g) \circ T$$
            für alle $ g \in G$ gilt.
        \item Die Menge aller $G$-Homomorphismen von 
            $V_ \rho$ nach $ V_ \pi$ bezeichnen wir mit
            $\Hom_G(V_ \rho, V_ \pi)$.
        \item Falls $ \rho$ und $ \pi$ unitäre Darstellungen
            sind, heißen diese \underline{unitär äquivalent}
            \index{unitär äquivalent}, falls es einen
            unitären $G$-\\Homomorphismus $T: V_ \rho\to 
            V_ \pi$ gibt.
    \end{enumerate}
\end{definition}
Wie man sich schon vorstellen kann, ist die Theorie
irreduzibler Darstellungen sehr überschaubar. Ein
zentrales Resultat dieser ist das sogenannte Lemma
von Schur:
\begin{lemma}[Schur]\label{lem:Schur}
Seien $( \rho, V_ \rho)$ eine unitäre Darstellung einer 
topologischen Gruppe $G$. Dann sind äquivalent:
\begin{enumerate}
    \item $ \rho$ ist irreduzibel.
    \item Falls $T \in \Hom_G( V_ \rho, V_ \rho)$, dann
        gilt $ T \in \C \id$.
\end{enumerate}
\end{lemma}
Für den Beweis dieses Lemma brauchen wir noch ein wenig 
Vorarbeit aus der Funktionalanalysis:
\begin{lemma}\label{lem:topologicalSchur}
Sei $H$ ein Hilbertraum und $A \subseteq \mathcal L(H)$ eine
Menge beschränkter Operatoren auf $H$, sodass für alle $S 
\in A$ auch $S^* \in A$ gilt. Dann sind äquivalent:
\begin{enumerate}
    \item Die einzige $A$-invariante, abgeschlossene Menge
        ungleich $ \{ 0 \} $ ist $ V$.
    \item Falls $T \in \mathcal L(H)$ mit allen $S \in A$
        kommutiert, dann ist $T = \lambda \id_V$ für
        ein $ \lambda \in \C$.
\end{enumerate}

\end{lemma}
\begin{proof}
Angenommen, (2) gilt. Dann sei $ \{ 0 \} \neq L \subseteq H$
ein abgeschlossener, $A$-invarianter Unterraum von $H$. Dann
ist auch $ L^\perp$ ein $A$-invarianter, abgeschlossener 
Untervektorraum, denn für alle $T \in A$, $x \in L^\perp$
und $y \in L$ gilt
$$ \langle Tx, y\rangle = \langle x, T^*y \rangle.$$
Da $T^* \in A$, folgt $T^* y \in L$. Somit gilt auch 
$$\langle x, T^*y \rangle = 0.$$
Da die gewählten Elemente beliebig waren, folgt $Tx \in 
L^\perp$ und somit ist $L^\perp$ ein $A$-invarianter,
abgeschlossener Untervektorraum. Dann betrachten wir
die orthogonale Projektion auf $L$ $P_L: H\to L$. Aus
obiger Rechnung sehen wir direkt mittels $H = L\oplus 
L^\perp$, dass diese mit allen $T \in A$ kommutiert.
Unter Ausnutzung von (2) erhalten wir also, dass 
$P_L = \id$ (da Projektionen Operatornorm 1 haben) gilt.
Somit muss $L= H$ gelten.\\
Angeommen (1) gilt. Sei also $T \in \mathcal L(H)$ ein 
Operator, der mit $A$ kommutiert. Dann kommutiert auch $T^*$
mit $A$, denn für $S \in A$ ist $S^* \in A$ ud somit
\begin{align*}
    \langle S T^* x, y\rangle &= \langle T^* x, S^* y
    \rangle\\
                                   &= \langle x, T S^*
                                      y\rangle\\
                                   &= \langle x, S^*T
                                   \rangle \\
                                   &= \langle T^* Sx, y
                                   \rangle,
\end{align*}
da $S^* \in A$. Da die Matrixkoeffizienten übereinstimmen,
folgt $S T^* = T^* S$. Indem wir $T= \frac{1}{2}(T+T^*) -
\frac{1}{2}i(iT - iT^*)$ betrachten, können wir ohne
Einschränkung annehmen, dass $T$ selbstadjungiert ist.
Da die Aussage für $T=0$ mit $ \lambda = 0$ klar ist,
können wir zudem $T\neq 0$ annehmen. Wir zeigen, dass
das Spektrum von $T$ aus einem einzigen Punkt
besteht. \\
Angenommen, es gibt $x\neq y \in \sigma(T)$. Dann gibt
es zwei Funktionen $f,g\in C( \sigma(T))$ mit
$f(x)\neq 0 \neq g(y)$ und $f\cdot g = 0$. Dann betrachten
wir das stetige Funktionalkalkül $f(T)$ und $g(T)$ und 
erinnern uns daran, dass jeder Operator $S$, der mit $T$ 
kommutiert, auch mit $f(T)$ kommutiert für 
$f\in C(\sigma(T))$.
Dann ist $f(T)\neq 0 \neq g(T)$ und $f\cdot g (T) = 0$,
da das holomorphe Funktionalkalkül multiplikativ ist.\\
Da $g(T)$ mit $A$ kommutiert, ist $L = \closure{g(T)H}$
ein $A$-invarianer Untervektorraum von $H$ ungleich dem 
Nullraum. Aus Anwendung von (1) erhalten wir, dass $L=H$.
Aber es gilt gleichzeitig auch, dass $ \{ 0 \} \neq f(T)H
= f(T) L \subseteq \closure{f(T) g(T) H} = \{ 0 \} $,
was einen Widerspruch darstellt. Somit muss das Spektrum
einelementig sein.
\end{proof}
Zusätzlich brauchen wir noch ein kleines Lemma, um ein
wenig mehr über unitäre Darstellungen zu erfahren.
\begin{lemma}\label{lem:unitaryReprMapsInversesToAdjoints}
Sei $( \rho, V)$ eine Darstellung. Dann sind äquivalent:
\begin{enumerate}
    \item $ \rho$ ist unitär.
    \item $ \rho( g^{-1}) = \rho(g)^*$ für alle $g \in G$.
\end{enumerate}
\end{lemma}
\begin{proof}
    \begin{align*}
        \rho \text{ ist unitär} &\iff \rho(g)^* \text{ 
        unitär }\forall g \in G\\
                                &\iff \rho(g)^{-1} = 
                                \rho(g)^* \forall g \in G\\
                                &\iff \rho( g^{-1}) = 
                                \rho(g)^* \forall g \in G,
    \end{align*}
    wobei im letzten Schritt verwendet wurde, dass $ \rho$
    ein Gruppenhomomorphismus ist.
\end{proof}
Nun können wir das Lemma \hyperref[lem:Schur]{Lemma von Schur} 
beweisen, welches mehr oder weniger ein direktes Korollar 
aus den obigen Lemmata \ref{lem:topologicalSchur} und
\ref{lem:unitaryReprMapsInversesToAdjoints} ist.
\begin{proof}[Beweis (Schur)]
    Nach Lemma \ref{lem:unitaryReprMapsInversesToAdjoints}
    ist $ A\coloneq\{ \rho(g): g \in G \} $ eine Menge, 
    für die $ S \in A \iff S^* \in A$ gilt. Daher können
    wir Lemma \ref{lem:topologicalSchur} anwenden und 
    erhalten die gewünschte Aussage.
\end{proof}
Während das Lemma von Schur im ersten Moment so wirkt,
als könne man damit nur die Homomorphismen einer Darstellung
auf sich selbst klassifizieren, liefert es auch viele
Informationen für die Homomorphismen zwischen beliebigen
irreduziblen Darstellungen.
\begin{korollar}
Seien $( \rho, V_ \rho), ( \pi, V_ \pi)$ zwei irreduzible,
unitäre Darstellungen. Dann gilt für einen $G$-
Homomorphismus $T: V_ \rho\to V_ \pi$, dass $T=0$ oder dass
$T$ invertierbar mit stetiger Umkehrabbildung ist. In diesem
Fall existiert ein $c>0$, sodass $ cT$ unitär ist. Dies
zeigt insbesondere, dass $\Hom_G(V_ \rho, V_ \pi) = \{ 0 \}$
gilt, außer $ \rho$ und $ \pi$ sind unitär äquivalent.
In diesem Fall ist $\dim(\Hom_G( V_ \rho, V_ \pi)) = 1$-
\end{korollar}
\begin{proof}
Sei $T: V_ \rho \to V_ \pi$ ein $G$-Homomorphismus. 
Betrachte $T^*: V_ \pi \to V_ \rho$. Diese Abbildung ist
ebenfalls ein $G$-Homomorphismus wie folgende Rechnung
zeigt:
\begin{align*}
    \langle \rho(g) T^* x, y \rangle &= \langle T^*x, 
    \rho(g)^* y \rangle\\ 
                                     &= \langle x, T \rho( 
                                     g^{-1}) y \rangle \\
                                     &= \langle x, 
                                 \pi(g^{-1})Ty\rangle\\
                                     &= \langle \pi(g)x, 
                                     Ty \rangle \\
                                     &= \langle T^* \pi(g)x
                                     ,y \rangle, 
\end{align*}
für alle $g \in G, x,y \in H$. Damit ist $T^* T \in 
\Hom_G(V_ \rho, V_ \rho)$. Aus dem \hyperref[lem:Schur]{
Lemma von Schur} folgt, dass $T^* T = \lambda \id$ für
ein $ \lambda \in \C$. Falls $T\neq 0$, dann ist
$T^* T \neq 0$. Durch die positive Semidefinitheit der 
Abbildung, folgt $ \lambda >0$. Betrachte $c = 
\sqrt{ \lambda^{-1}}$. Dann ist $(cT)^*(cT) = \id$. 
Zusätzlich folgt analog, dass $T T^*$ bijektiv ist. Damit
muss $cT$ bijektiv sein und somit auch unitär.\\
Für je zwei $G$-Homomorphismen $S,T: V_ \rho\to V_ \pi$ mit
$T\neq 0$ folgt also, dass $S \circ T^{-1} \in \Hom_G(V_ 
\pi, V_ \pi)$. Somit $S \circ T^{-1} = \lambda \id$ für
ein $ \lambda \in\C$. Somit $S = \lambda T$. Dies zeigt
die Eindimensionalität.
\end{proof}
\printindex
\end{document}
