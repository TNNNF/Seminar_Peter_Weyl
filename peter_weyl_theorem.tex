\documentclass{report}

\usepackage{style}
\addbibresource{references.bib}

\begin{document}
\tableofcontents
\chapter{Einführung}
\section{Motivation}
Das zentrale Ergebnis des letzten Vortrags war die Existenz
und Eindeutigkeit des Haarmaßes, welche sich im folgenden
Satz zusammenfassen lässt.
\begin{satz*}[Existenz und Eindeutigkeit des Haarmaßes]
    Sei $G$ eine lokalkompakte Gruppe. Dann gibt es eine 
    linksinvariantes Radonmaß $ \mu\neq 0$ auf 
    $\mathcal{B}(X)$. Dieses ist eindeutig bis auf 
    positive Vielfache.
\end{satz*}
Dieses Maß wird als das Haarmaß bezeichnet. Somit erhalten
wir auf jeder Gruppe einen sinnvollen Begriff des Integrals.
Dies erlaubt uns, wirklich sinnvoll über Räume wie 
$\LL^2(G)$ zu reden.\\
Im Spezialfall erhalten wir dann natürlich Ergebnisse für
die Integrationstheorie auf dem $\R^d$ oder anderen 
spannenden Gruppen wie $\R^\times$. Jedoch sind nicht alle
Gruppen so gut strukturiert wie diese Gruppen. Diese sind
nämlich abelsch und lokalkompakt. Die Integrationstheorie
auf solchen Gruppen ist zentraler Gegenstand der 
\emph{harmonischen Analysis}. Der nicht abelsche Fall ist
jedoch nicht so gut strukturiert und deutlich 
unübersichtlicher. Für den kompakten Fall erhalten wir 
jedoch den \emph{Satz von Peter-Weyl}, welcher viele 
Informationen über die Struktur von $\LL^2(G)$ beinhaltet.
Diesen zu verstehen und zu beweisen ist Gegenstand des 
vorliegenden Seminarvortrags.\\
Zunächst brauchen wir jedoch noch Grundlagen der
Darstellungstheorie topologischer Gruppen. Die Primärquelle
dieses Vortrags ist \cite{principlesHarmAna}. An einigen 
Stellen greifen wir auf \cite{reprCompLieGroups} zurück.
\section{Notation}
\begin{itemize}
    \item Für einen Banachraum $V$ bezeichnen wir mit 
        $\GLS{V}$ die Menge der stetigen, linearen, 
        invertierbaren Abbildungen.
    \item Für zwei Banachräume $V, W$ bezeichnen wir mit
        $ \mathcal L(V, W)$ die Menge der beschränkten,
        stetigen linearen Abbildungen $T\colon V\to W$.
    \item Für einen Maßraum $( \Omega, \mathcal{A}, \mu)$
	    bezeichnen wir mit $\LL^2( \Omega)$ den
	    üblichen Raum der quadratintegrierbaren
	    Funktionen $f\colon \Omega\to \C$. Auf diesem
	    ist das übliche Skalarprodukt gegeben.
	    Dieses wird mit $ \langle \cdot, 
	    \cdot \rangle_{\LL^2}$ notiert.
\end{itemize}
\chapter{Grundlagen}
\section{Haarsche Integrationstheorie}
Sei in diesem Kapitel stets $G$ eine lokalkompakte, topologische
Gruppe mit zugehörigem Haarmaß $ \mu$. Dann gelten folgende
Eigenschaften:
\begin{satz}[Eigenschaften des Haarmaßes]
    \label{thm:propsHaarMeasure}
In obiger Situation gelten die folgenden Eigenschaften:
\begin{enumerate}
    \item Sei $U$ eine nichtleere, offene Menge. Dann gilt
        $ \mu(U)>0$.
    \item Sei $f$ eine stetige Funktion mit $f \geq 0$ und
        $ \int_{G} f(x)  \,\diff \mu(x) =0 $. Dann gilt
        $ f = 0$.
\end{enumerate}
\end{satz}
\begin{proof}
\begin{enumerate}
    \item Sei $U\neq \emptyset$ offen. Dann sind die Mengen
        $xU$ jeweils offen für alle $ x \in G$ und überdecken
        die gesamte Gruppe $G$. Insbesondere überdecken
        sie also alle kompakten Mengen endlich. Da $ \mu$
        ein Haarmaß, also insbesondere translationsinvariant
        ist, folgt, dass $ \mu(xU) = 0$ für alle $x \in G$.
        Somit folgt, dass alle kompakten Mengen durch
        obige endliche Überdeckung der $xU$ ebenfalls
        Maß $0$ haben. Da $ \mu$ ein Radonmaß ist, folgt, dass
        $ \mu = 0$, was ein Widerspruch dazu ist, dass $ \mu$
        ein Haarmaß ist. Somit ist $ \mu(U)> 0$.
    \item Sei $f$ eine solche Funktion. Dann betrachten wir
        $ U\coloneq f^{-1}(0, \infty)$. Da $f \geq 0$ und $f$ 
        stetig ist, muss direkt folgen, dass $ U = \emptyset$. 
        Denn wenn $ \mu(U) >0 $ gelten würde, müsste durch die
        Stetigkeit auf $f$ das Integral größer 0 sein. Das
        führt zum Widerspruch zur Annahme. Somit $ \mu(U)=0$
        und damit $ f = 0$ nach der vorherigen Aussage.
\end{enumerate}

\end{proof}

\section{Funktionalanylsis}
\chapter{Darstellungstheorie}
Ziel der Darstellungstheorie ist es gewissermaßen, das 
Problem der schwierigen Strukturierung von Gruppen zu
lösen, indem man die Elemente in lineare Operatoren 
übersetzt.
Für diese kennen wir nämlich aus der linearen Algebra im 
endlichdimensionalen sowie der Funktionalanalysis um 
unendlichdimensionalen Fall viele hilfreiche Sätze, mit 
denen wir sehr häufig mehr über die Struktur der Gruppe 
verstehen können.
\section{Grundlegende Begriffe}
Der zentrale Begriff in der Darstellungstheorie 
topologischer Gruppen ist der einer Darstellung.
\begin{definition}[Darstellung]
Sei $V$ ein Banachraum und $G$ eine topologische Gruppe. 
Eine \underline{Banachraum-Darstellung}
\index{Banachraum-Darstellung}\footnote{Häufig werden wir 
nur Darstellung sagen und damit eine Banachraum-Darstellung 
meinen.} von $G$ auf $V$ ist ein Gruppenhomomorphismus 
$ \pi\colon G\to \GLS{V}$, sodass die Abbildung $$G\times V\to V,
(g,v)\mapsto (g,v)\mapsto \pi(g)v$$ stetig ist.
\end{definition}
\begin{beispiel}[Linksreguläre Darstellung]
    \label{ex:leftRegularRepr}
Jede lokalkompakte Gruppe $G$ wirkt auf dem Hilbertraum
ihrer quadratintegrierbaren Funktionen $ \LL^2(G)$ mit
der sogenannten \underline{linksregulären Wirkung}
\index{linksreguläre Wirkung}, welche durch 
$$ \rho_L\colon G\to \GLS{\LL^2(G)}, x\mapsto L_x$$
gegeben wird mit $ L_x\colon \LL^2(G)\to \LL^2(G), \phi 
\mapsto \phi \circ l_{ x^{-1}}$, wobei $l_x$ die 
Linksmultiplikation mit einem Element $x$ bezeichnet.
\end{beispiel}

Bereits aus der Algebra 1 wissen wir um die Wichtigkeit 
bestimmter "elementarer" Strukturen, z.B. den zyklischen
Gruppen bei der Klassifikation endlicher abelscher Gruppen.
Auch in der Darstellungstheorie gibt es einen ähnlichen
Begriff:
\begin{definition}[Unterdarstellungen, irreduzible 
    Darstellungen]
Sei $( \rho, V_ \rho)$ eine Darstellung einer topologischen
    Gruppe $G$. \begin{enumerate}
    \item $( \pi, V_ \pi)$ heißt \underline{
        Unterdarstellung}\index{Unterdarstellung} von 
        $ ( \rho, V_ \rho)$, falls $ V_ \pi$ ein 
        abgeschlossener Untervektorraum von $ V_ \rho$
        ist und $ \rho$ eingeschränkt auf $ V_ \pi$
        gleich $ \pi$ ist.
    \item $( \rho, V_ \rho)$ heißt \underline{irreduzibel}
        \index{irreduzibel}, falls es keine nicht-trivialen
        Unterdarstellungen gibt, also falls für jede 
        Unterdarstellung $( \pi, V_ \pi)$ gilt, dass
        $V_ \pi = \{0\}$ oder $ V_ \pi = V_ \rho$.
\end{enumerate}
\end{definition}
\begin{bemerkung}
Jeder abgeschlossene Untervektorraum $U \subseteq V_ \rho$, 
für den $ \rho(G) U \subseteq U$ gilt, liefert eine 
Unterdarstellung durch Einschränken von $ \rho$ auf $ U$.
\end{bemerkung}

Für den Fall, dass die Banachräume $V$ für unsere 
Darstellungen sogar Hilberträume sind, gibt es noch
eine weitere Art von Darstellungen:
\begin{definition}[Unitäre Darstellungen]
Sei $V$ ein Hilbertraum und $G$ eine topologische Gruppe.
Eine Darstellung $ \rho$ auf $ V$ heißt \underline{unitäre 
Darstellung}\index{unitäre Darstellung}, falls 
$ \rho(g)$ unitär ist für alle $g \in G$, also
$ \langle \rho(g)v, \rho(g)w \rangle = \langle v, w\rangle$
für alle $ g \in G$ und alle $ v,w \in V$.
\end{definition}
Wir möchten später zeigen, dass sich jede endlichdimensionale
Darstellung als direkte Summe von irreduziblen Darstellungen
schreiben lässt. Dazu definieren wir
\begin{definition}[Direkte Summe von Darstellungen]
Seien $( \rho_1, V_1), (\rho_2, V_2)$ zwei unitäre 
Darstellungen. Auf der direkten Summe von Vektorräumen
$V \coloneq V_1 \oplus V_2$ definieren wir die
\underline{direkte Summe von Darstellungen}
\index{direkte Summe von Darstellungen} $\rho = 
\rho_1 \oplus \rho_2$ durch komponentenweise Wirkung. 
Analog definiert man dies für beliebige Indexfamilien
$I$.
\end{definition}

Besonders relevant werden auch Beziehungen zwischen
Darstellungen, die wie üblich über Abbildungen gegeben sind.
\begin{definition}[ $G$-Homomorphismen, Äquivalenz von 
    Darstellungen]
    Seien $( \rho, V_ \rho), ( \pi, V_ \pi)$ zwei
    Darstellungen von $G$.
    \begin{enumerate}
        \item Ein stetiger, linearer Operator $T\colon V_ \rho 
            \to V_ \pi$ heißt \underline{$G$-Homomorphismus}
            \index{$G$-Homomorphismus}, falls
            $$ T \circ \rho(g) = \pi(g) \circ T$$
            für alle $ g \in G$ gilt.
        \item Die Menge aller $G$-Homomorphismen von 
            $V_ \rho$ nach $ V_ \pi$ bezeichnen wir mit
            $\Hom_G(V_ \rho, V_ \pi)$.
        \item Falls $ \rho$ und $ \pi$ unitäre Darstellungen
            sind, heißen diese \underline{unitär äquivalent}
            \index{unitär äquivalent}, falls es einen
            unitären $G$-\\Homomorphismus $T\colon V_ \rho\to 
            V_ \pi$ gibt.
    \end{enumerate}
\end{definition}
\section{Lemma von Schur}
Wie man sich schon vorstellen kann, ist die Theorie
irreduzibler Darstellungen sehr überschaubar. Ein
zentrales Resultat dieser ist das sogenannte Lemma
von Schur:
\begin{lemma}[Schur]\label{lem:Schur}
Seien $( \rho, V_ \rho)$ eine unitäre Darstellung einer 
topologischen Gruppe $G$. Dann sind äquivalent:
\begin{enumerate}
    \item $ \rho$ ist irreduzibel.
    \item Falls $T \in \Hom_G( V_ \rho, V_ \rho)$, dann
        gilt $ T \in \C \id$.
\end{enumerate}
\end{lemma}
Für den Beweis dieses Lemma brauchen wir noch ein wenig 
Vorarbeit aus der Funktionalanalysis:
\begin{lemma}\label{lem:topologicalSchur}
Sei $H$ ein Hilbertraum und $A \subseteq \mathcal L(H)$ eine
Menge beschränkter Operatoren auf $H$, sodass für alle $S 
\in A$ auch $S^* \in A$ gilt. Dann sind äquivalent:
\begin{enumerate}
    \item Die einzige $A$-invariante, abgeschlossene Menge
        ungleich $ \{ 0 \} $ ist $ H$.
    \item Falls $T \in \mathcal L(H)$ mit allen $S \in A$
        kommutiert, dann ist $T = \lambda \id_V$ für
        ein $ \lambda \in \C$.
\end{enumerate}

\end{lemma}
\begin{proof}
Angenommen, (2) gilt. Dann sei $ \{ 0 \} \neq L \subseteq H$
ein abgeschlossener, $A$-invarianter Unterraum von $H$. Dann
ist auch $ L^\perp$ ein $A$-invarianter, abgeschlossener 
Untervektorraum, denn für alle $T \in A$, $x \in L^\perp$
und $y \in L$ gilt
$$ \langle Tx, y\rangle = \langle x, T^*y \rangle.$$
Da $T^* \in A$, folgt $T^* y \in L$. Somit gilt auch 
$$\langle x, T^*y \rangle = 0.$$
Da die gewählten Elemente beliebig waren, folgt $Tx \in 
L^\perp$ und somit ist $L^\perp$ ein $A$-invarianter,
abgeschlossener Untervektorraum. Dann betrachten wir
die orthogonale Projektion auf $L$ $P_L\colon H\to L$. Aus
obiger Rechnung sehen wir direkt mittels $H = L\oplus 
L^\perp$, dass diese mit allen $T \in A$ kommutiert.
Unter Ausnutzung von (2) erhalten wir also, dass 
$P_L = \id$ (da Projektionen Operatornorm 1 haben) gilt.
Somit muss $L= H$ gelten.\\
Angeommen (1) gilt. Sei also $T \in \mathcal L(H)$ ein 
Operator, der mit $A$ kommutiert. Dann kommutiert auch $T^*$
mit $A$, denn für $S \in A$ ist $S^* \in A$ ud somit
\begin{align*}
    \langle S T^* x, y\rangle &= \langle T^* x, S^* y
    \rangle\\
                                   &= \langle x, T S^*
                                      y\rangle\\
                                   &= \langle x, S^*T
                                   \rangle \\
                                   &= \langle T^* Sx, y
                                   \rangle,
\end{align*}
da $S^* \in A$. Da die Matrixkoeffizienten übereinstimmen,
folgt $S T^* = T^* S$. Indem wir $T= \frac{1}{2}(T+T^*) -
\frac{1}{2}i(iT - iT^*)$ betrachten, können wir ohne
Einschränkung annehmen, dass $T$ selbstadjungiert ist.
Da die Aussage für $T=0$ mit $ \lambda = 0$ klar ist,
können wir zudem $T\neq 0$ annehmen. Wir zeigen, dass
das Spektrum von $T$ aus einem einzigen Punkt
besteht. \\
Angenommen, es gibt $x\neq y \in \sigma(T)$. Dann gibt
es zwei Funktionen $f,g\in C( \sigma(T))$ mit
$f(x)\neq 0 \neq g(y)$ und $f\cdot g = 0$. Dann betrachten
wir das stetige Funktionalkalkül $f(T)$ und $g(T)$ und 
erinnern uns daran, dass jeder Operator $S$, der mit $T$ 
kommutiert, auch mit $f(T)$ kommutiert für 
$f\in C(\sigma(T))$.
Dann ist $f(T)\neq 0 \neq g(T)$ und $f\cdot g (T) = 0$,
da das holomorphe Funktionalkalkül multiplikativ ist.\\
Da $g(T)$ mit $A$ kommutiert, ist $L = \closure{g(T)H}$
ein $A$-invarianer Untervektorraum von $H$ ungleich dem 
Nullraum. Aus Anwendung von (1) erhalten wir, dass $L=H$.
Aber es gilt gleichzeitig auch, dass $ \{ 0 \} \neq f(T)H
= f(T) L \subseteq \closure{f(T) g(T) H} = \{ 0 \} $,
was einen Widerspruch darstellt. Somit muss das Spektrum
einelementig sein.
\end{proof}
Zusätzlich brauchen wir noch ein kleines Lemma, um ein
wenig mehr über unitäre Darstellungen zu erfahren.
\begin{lemma}\label{lem:unitaryReprMapsInversesToAdjoints}
Sei $( \rho, V)$ eine Darstellung. Dann sind äquivalent:
\begin{enumerate}
    \item $ \rho$ ist unitär.
    \item $ \rho( g^{-1}) = \rho(g)^*$ für alle $g \in G$.
\end{enumerate}
\end{lemma}
\begin{proof}
    \begin{align*}
        \rho \text{ ist unitär} &\iff \rho(g)^* \text{ 
        unitär }\forall g \in G\\
                                &\iff \rho(g)^{-1} = 
                                \rho(g)^* \forall g \in G\\
                                &\iff \rho( g^{-1}) = 
                                \rho(g)^* \forall g \in G,
    \end{align*}
    wobei im letzten Schritt verwendet wurde, dass $ \rho$
    ein Gruppenhomomorphismus ist.
\end{proof}
Nun können wir das Lemma \hyperref[lem:Schur]{Lemma von Schur} 
beweisen, welches mehr oder weniger ein direktes Korollar 
aus den obigen Lemmata \ref{lem:topologicalSchur} und
\ref{lem:unitaryReprMapsInversesToAdjoints} ist.
\begin{proof}[Beweis (Schur)]
    Nach Lemma \ref{lem:unitaryReprMapsInversesToAdjoints}
    ist $ A\coloneq\{ \rho(g): g \in G \} $ eine Menge, 
    für die $ S \in A \iff S^* \in A$ gilt. Daher können
    wir Lemma \ref{lem:topologicalSchur} anwenden und 
    erhalten die gewünschte Aussage.
\end{proof}
Während das Lemma von Schur im ersten Moment so wirkt,
als könne man damit nur die Homomorphismen einer Darstellung
auf sich selbst klassifizieren, liefert es auch viele
Informationen für die Homomorphismen zwischen beliebigen
irreduziblen Darstellungen.
\begin{korollar}\label{cor:HomSpaceOneDimensional}
Seien $( \rho, V_ \rho), ( \pi, V_ \pi)$ zwei irreduzible,
unitäre Darstellungen. Dann gilt für einen $G$-
Homomorphismus $T\colon V_ \rho\to V_ \pi$, dass $T=0$ oder dass
$T$ invertierbar mit stetiger Umkehrabbildung ist. In diesem
Fall existiert ein $c>0$, sodass $ cT$ unitär ist. Dies
zeigt insbesondere, dass $\Hom_G(V_ \rho, V_ \pi) = \{ 0 \}$
gilt, außer $ \rho$ und $ \pi$ sind unitär äquivalent.
In diesem Fall ist $\dim(\Hom_G( V_ \rho, V_ \pi)) = 1$-
\end{korollar}
\begin{proof}
Sei $T\colon V_ \rho \to V_ \pi$ ein $G$-Homomorphismus. 
Betrachte $T^*\colon V_ \pi \to V_ \rho$. Diese Abbildung ist
ebenfalls ein $G$-Homomorphismus wie folgende Rechnung
zeigt:
\begin{align*}
    \langle \rho(g) T^* x, y \rangle &= \langle T^*x, 
    \rho(g)^* y \rangle\\ 
                                     &= \langle x, T \rho( 
                                     g^{-1}) y \rangle \\
                                     &= \langle x, 
                                 \pi(g^{-1})Ty\rangle\\
                                     &= \langle \pi(g)x, 
                                     Ty \rangle \\
                                     &= \langle T^* \pi(g)x
                                     ,y \rangle, 
\end{align*}
für alle $g \in G, x,y \in H$. Damit ist $T^* T \in 
\Hom_G(V_ \rho, V_ \rho)$. Aus dem \hyperref[lem:Schur]{
Lemma von Schur} folgt, dass $T^* T = \lambda \id$ für
ein $ \lambda \in \C$. Falls $T\neq 0$, dann ist
$T^* T \neq 0$. Durch die positive Semidefinitheit der 
Abbildung, folgt $ \lambda >0$. Betrachte $c = 
\sqrt{ \lambda^{-1}}$. Dann ist $(cT)^*(cT) = \id$. 
Zusätzlich folgt analog, dass $T T^*$ bijektiv ist. Damit
muss $cT$ bijektiv sein und somit auch unitär.\\
Für je zwei $G$-Homomorphismen $S,T\colon V_ \rho\to V_ \pi$ mit
$T\neq 0$ folgt also, dass $S \circ T^{-1} \in \Hom_G(V_ 
\pi, V_ \pi)$. Somit $S \circ T^{-1} = \lambda \id$ für
ein $ \lambda \in\C$. Somit $S = \lambda T$. Dies zeigt
die Eindimensionalität.
\end{proof}
\section{Kompakte Gruppen}
Das Lemma von Schur liefert uns schon eine gute
Charakterisierung irreduzibler Darstellungen. Jedoch
interessieren uns auch nicht notwendigerweise irreduzible
Darstellungen. Diese können im Allgemeinen etwas
unübersichtlicher werden. Für kompakte Gruppen wird die 
Situation jedoch recht überschaubar und hier wird uns das
erste Mal die Integration bezüglich des Haarmaßes gute 
Dienste erweisen. \\
Zuerst beginnen wir damit, die endlichdimensionalen 
Darstellungen genauer zu studieren. Dies wirkt auf den 
ersten Blick stark einschränkend. Wir werden jedoch sehen,
wie mächtig diese Darstellungen auf kompakten Gruppen 
sind.\\
Bevor wir damit jedoch starten können, müssen wir zumindest
kurz über die Haarmaße auf kompakten Gruppen reden. A
priori ist nämlich nicht klar, ob es einen Unterschied 
zwischen linken und rechten Haarmaßen gibt oder nicht.
Dazu brauchen wir einige Hilfsmittel.
\begin{definition}[modulare Funktion, unimodulare Gruppen]
    Sei $G$ eine lokalkompakte Gruppe und $ \mu$ ein
    Haarmaß auf $G$. Für $x \in G$ definieren wir das Maß
    $ \mu_x$ durch $ \mu_x(A) = \mu(Ax)$, wobei es sich
    ebenfalls um ein Haarmaß handelt. Somit gibt es durch
    die Eindeutigkeit des Haarmaßes eine Zahl $\Delta(x)>0$,
    mit $ \mu_x = \Delta(x) \mu$. Dies definiert eine 
    Abbildung $ \Delta\colon G\to \R_{>0}$, welche die
    \underline{modulare Funktion}\index{modulare Funktion}
    der Gruppe $G$ genannt wird.\\
    Wenn $ \Delta \equiv 1$, dann heißt $G$ eine
    \underline{unimodulare Gruppe}\index{unimodulare Gruppe}
    .
\end{definition}
\begin{bemerkung}
Für unimodulare Gruppen stimmen linke und rechte Haarmaße
überein.
\end{bemerkung}
Damit wir uns keine Gedanken über das Maß machen müssen,
wollen wir zeigen, dass kompakte Gruppen unimodular sind.
\begin{satz}\label{thm:propsModularFunction}
Sei $G$ eine lokalkompakte Gruppe mit modularer Funktion
$ \Delta\colon G\to \R^\times_{>0}$. Dann gelten folgende 
Eigenschaften:
\begin{enumerate}
    \item Für $ y \in G$ und $ f \in \LL^1(G)$ ist
        $R_y f \in \LL^1(G)$ und es gilt
        $$ \int_{G} R_y f(x) \,\diff x = \int_{G} f(xy) 
        \,\diff x = \Delta( y^{-1}) \int_{G} f(x) 
        \,\diff x,$$
        wobei $R_yf\colon G\to \R, x \mapsto f(xy)$.
    \item $ \Delta$ ist ein stetiger Gruppenhomomorphismus.
    \item Kompakte Gruppen sind unimodular.
\end{enumerate}
\end{satz}
\begin{proof}
    \begin{enumerate}
        \item Für charakteristische Funktionen 
            $\mathds{1}_A$ gilt:
            \begin{align*}
                \int_{_G} \mathds{1}_A(xy) \,\diff x &=
                \int_{A} xy \,\diff x \\
                &= \int_{Ay^{-1}} x \,\diff  x\\
                &= \mu(A y^{-1})\\
                &= \Delta( y^{-1}) \mu(A)\\
                &= \Delta( y^{-1}) \int_{G} \mathds{1}_A(x)
                \,\diff x.
            \end{align*}
            Für beliebige integrierbare Funktionen folgt die
            Aussage mit maßtheoretischer Induktion.
        \item Für $x,y \in G$ und eine messbare Menge
            $A \subseteq G$ gilt
            \begin{align*}
                \Delta(xy) \mu(A) &= \mu_{xy}(A) \\
                                  &= \mu(Axy)\\
                                  &= \mu_y(Ax)\\
                                  &= \Delta(y) \mu(Ax)\\
                                  &= \Delta(y) \Delta(x) 
                                  \mu(A).
            \end{align*}
            Durch Wahl einer beliebigen Menge $A$ mit
            $ 0 < \mu(A) < \infty$ gilt dann
            $$ \Delta(xy) = \Delta(x) \Delta(y).$$
            Zur Stetigkeit in $y$ wähle eine Funktion
            $f \in C_c(G)$ mit $ c = \int_{G} f(x)  \,\diff
            x > \neq 0$. Dann folgt durch den ersten Teil
            der Aussage 
            $$ \Delta(y) = \frac{1}{c} \int_{G} f(x y^{-1}) 
            \,\diff x  = \frac{1}{c} \int_{G} R_{ y^{-1}} 
            f(x) \,\diff  x.$$
            Dieser Ausdruck ist stetig in $y$, weswegen auch
            $ \Delta$ stetig in $y$ sein muss.
        \item Sei nun $G$ kompakt. Da $ \Delta$ ein
            stetiger Gruppenhomomorphismus ist, muss
            $ \Delta(G)$ eine kompakte Untergruppe von
            $ \R_{>0}^\times$ sein. Die einzige kompakte
            Untergruppe dieser Gruppe ist jedoch $ \{ 1 \}$.
    \end{enumerate} 
\end{proof}
Dass wir nicht zwischen linken und rechten Haarmaßen
unterscheiden müssen, erleichtert die Situation erheblich,
weswegen wir uns nun den endlichdimensionalen Darstellungen
widmen können. Sei also im Folgenden stets $K$ eine kompakte
(also insbesondere lokalkompakte) topologische Gruppe sowie
$ (\rho, V)$ eine endlichdimensionale Darstellung, sofern
nicht anders gefordert.
\begin{lemma}
\label{lem:finiteDimensionalRepresentationsAlwaysUnitary}
Auf $V$ gibt es stets ein Skalarprodukt, sodass $ \rho$
eine unitäre Darstellung wird. Falls $ \rho$ irreduzibel
ist, dann ist dieses Skalarprodukt eindeutig bis auf
eine positive Konstante bestimmt.
\end{lemma}
\begin{proof}
Wir wählen ein beliebiges Skalarprodukt $( \cdot, \cdot)\colon
V\times V\to \C$. Dann definieren wir ein neues 
Skalarprodukt durch den Ausdruck
$$ \langle v, w \rangle \coloneq \int_{K} (
 \rho(k)v, \rho(k)w)\,\diff k$$
bezüglich des Haarmaßes $ \mu$, für das $ \mu(K) = 1$ gilt.
Wir zeigen nun, dass dies tatsächlich ein Skalarprodukt
definiert:
\begin{enumerate}
    \item Die Linearität folgt direkt aus der Linearität
        von $ (\cdot, \cdot)$ und des Integrals.
    \item Dass das Skalarprodukt hermitesch ist, folgt
        unmittelbar aus der Hermitizität des ursprünglichen
        Skalarproduktes und dem Fakt, dass Konjugieren und
        Integrieren kommutieren.
    \item Es bleibt nur noch zu zeigen, dass die so
        definierte Abbildung positiv definit ist. Dabei ist
        die Positivität ebenfalls klar. Sei also $ v \in V$,
        sodass $ \langle v,v \rangle =0$. Dann gilt
        $$ 0 = \langle v,v \rangle = \int_{K} ( \tau(k)v, 
        \tau(k)v)  \,\diff k.$$
        Die Abbbildung $ k \mapsto (\tau(k)v, 
        \tau(k)v) $ ist stetig und somit können
        wir obigen Satz \ref{thm:propsHaarMeasure}
        verwenden und erhalten, dass $ k \mapsto ( \tau(k)v
        , \tau(k)v)$ gleich der Nullabbildung ist. 
        Insbesondere gilt also für $k = 1_G$, dass
        $ \tau(1_G) = \id_V$. Somit folgt also ebenfalls
        $ (v, v) = 0$. Da $(\cdot, \cdot)$ ein Skalarprodukt
        ist, folgt $ v = 0$.
\end{enumerate}
Damit ist $ \langle \cdot, \cdot \rangle $ ein Skalarprodukt
und wir müssen nur noch zeigen, dass $ \rho$ unitär 
bezüglich diesen Skalarprodukts ist. Dazu sehen wir, dass
\begin{align*}
    \langle \tau(x)v, \tau(x)w \rangle 
    &= \int_{K}(\tau(k)\tau(x)v, \tau(k)\tau(x)w)\,\diff k\\
    &= \int_{K}( \tau(kx)v, \tau(kx)w) \,\diff  k\\
    &= \int_{K} ( \tau(k)v, \tau(k)w) \,\diff k\\
    &= \langle v,w \rangle, 
\end{align*}
da die Gruppe $K$ unimodular ist und es sich mit $f\colon
K\to \C, k\mapsto ( \tau(k)v, \tau(k)w)$ bei
$R_x f \colon K\to \C$ genau um $k \mapsto ( \tau(kx)v, 
\tau(kx)w)$ handelt und wir somit 
Satz \ref{thm:propsModularFunction} anwenden können. Was
noch zu zeigen ist, ist die Eindeutigkeit bis auf 
positive Konstanten im Fall irreduzibler Darstellungen. 
Dazu seien $( \rho_1, V_1), (\rho_2, V_2)$ Darstellungen,
die sich nur im unitären Skalarprodukt auf $V$ 
unterscheiden. Bezeichne diese Skalarprodukte mit 
$ \langle \cdot, \cdot \rangle_1$ und 
$ \langle \cdot,\cdot \rangle_2$. Betrachte nun die
lineare Abbildung $\id\colon V_1\to V_2, v\mapsto v$. Da
$V_1$ und $V_2$ endlichdimensional ist, ist $\id$ 
stetig und ein $G$-Homomorphismus zwischen $ \rho_1$
und $\rho_2$. Nun nutzen wir Korollar
\ref{cor:HomSpaceOneDimensional}, um zu folgern, dass
es ein $c>0$ gibt, sodass $c\cdot \id$ unitär zwischen
den beiden Darstellungen ist. Insbesondere folgt dann
also, dass 
\begin{align*}
	c^2 \langle v_1, v_2 \rangle_2 
	&= \langle c v_1, c v_2 \rangle_2 \\
	&= \langle v_1, v_2 \rangle_1
\end{align*}
Damit sind die Skalarprodukte bis auf positive Konstante
gleich.
\end{proof}
Zusätzlich lassen sich die endlichdimensionalen 
Darstellungen gut in "Atome" -- die irreduziblen 
Darstellungen -- zerlegen.
\begin{lemma}\label{lem:FiniteDimReprSumOfIrredRepr}
Sei $( \rho, V)$ eine endlichdimensionale Darstellung 
von $K$. Dann gibt es irreduzible Darstellungen 
$(\rho_i, V_i)_{i \in I}$, sodass $$ \rho = 
\bigoplus_ { i \in I } \rho_i$$.
\end{lemma}
\begin{proof}
Sei $K$ eine kompakte, topologische Gruppe und $(\rho, V)$
eine Darstellung von $K$. Da $V$ endlichdimensional ist,
können wir die Aussage mit Induktion über $\dim(V)$
führen.\\
\underline{I.A. ($\dim(V)=1$):} Dieser Fall ist klar, da die
einzigen Unterräume von $V$ $ \{ 0 \} $ und $V$ sind.
Somit muss jede eindimensionale Darstellung schon
direkt irreduzibel sein.\\
\underline{I.S.:} Sei $ V$ ein Vektorraum der Dimension
$ n+1$ für ein beliebiges $ n \in \N_0$. Angenommen, wir 
haben die Aussage schon für alle Vektorräume der 
Dimension kleiner gleich $n$ gezeigt. Zusätzlich können
wir mit Lemma 
\ref{lem:finiteDimensionalRepresentationsAlwaysUnitary}
annehmen, dass $ \rho$ eine unitäre Darstellung ist. 
Falls $\rho$ bereits irreduzbiel ist, sind wir fertig.
Ansonsten gibt es einen $\rho(K)$-invarianten 
Untervektorraum $ \{ 0 \} \subsetneq  U \subsetneq V$. 
Betrachte dessen orthogonales Komplement $U^\perp$. 
Auch dieser ist abgeschlossen unter $\rho(K)$, denn 
für $ g \in K, u \in U, v \in U^\perp$ gilt:
\begin{align*}
	\langle u, \rho(g)v \rangle 
	&= \langle \rho(g)^*u, v \rangle \\
	&= \langle \rho(g^{-1})u, v \rangle\\
	&= 0,
\end{align*}
wobei wir im vorletzten Schritt verwendet haben,
dass $ \rho(g)$ unitär ist, und im letzten
Schritt, dass $U$ $\rho(K)$-invariant ist.
Nun können wir die Induktionsvoraussetzung auf
$U$ und $U^\perp$ anwenden und erhalten durch
$ V = U \oplus U^\perp$ die gewünschte Aussage.
\end{proof}
Wir beenden diesen Abschnitt mit zwei Definitionen
und einem Korollar:
\begin{definition}[unitäres Dual, endliches unitäres Dual]
Sei $G$ eine lokalkompakte Gruppe. Dann bezeichnen
wir mit $\UDual{G}$ die Menge aller Äquivalenzklassen
irreduzibler Darstellungen auf $G$ bezüglich
unitärer Äquivalenz. Wir nennen $\UDual{G}$
\underline{unitäres Dual}\index{unitäres Dual}
zu $G$.\\
Zusätzlich bezeichnen wir für eine kompakte Gruppe $K$
die Äquivalenzklassen der endlichdimensionalen,
irreduziblen Darstellungen mit $\UDualFin{K}$ und
nennen dies das \underline{endliche, unitäre Dual}
\index{endliche, unitäre Dual} von $K$.
Darstellungen
\end{definition}
\begin{bemerkung}
A priori gibt es ein kleines mengentheoretisches Problem
bei der Definition dieser beiden Mengen, da nicht trivial
ist, dass es sich wirklich um Mengen handelt. Um dies zu
beheben, müssten wir ein wenig in die Theorie der 
Kardinalzahlen eintauchen. Dies wollen wir hier nicht
weiter vertiefen.
\end{bemerkung}
Ein wesentliches Ziel des Vortrags ist es zu zeigen,
dass $\UDual{K} = \UDualFin{K}$ gilt.
\begin{korollar}
	Es gilt stets $\UDualFin{K}\subseteq \UDual{K}$.
\end{korollar}
\begin{proof}
Zu zeigen ist bloß, dass alle endlichdimensionalen
Darstellungen auch unitär sind. Dies haben wir 
bereits in Lemma 
\ref{lem:finiteDimensionalRepresentationsAlwaysUnitary}
gezeigt.
\end{proof}
\chapter{Der Satz von Peter-Weyl}
\section{Orthogonalitätseigenschaften}
Zuerst wollen wir eine gewisse Verallgemeinerung von
Matrixelementen auf unitäre Darstellungen definieren. 
\begin{definition}[Matrixkoeffizienten]
Sei $K$ eine kompakte, topologische Gruppe und $( \tau, V)$
eine unitäre Darstellung von $K$. Die 
\underline{Matrixkoeffizienten}\index{Matrixkoeffizienten}
von $ \tau$ sind Abbildungen der Form $ k\mapsto 
\langle \tau(k)v, w \rangle $ für beliebige $v,w \in V$.
\end{definition}
Für die Matrixkoeffizienten können wir einige spannende
Eigenschaften feststellen:
\begin{lemma}\label{lem:propsMatrixCoeffs}
Sei $K$ eine kompakte, topologische Gruppe und $( \tau, V)$
eine endlichdimensionale Darstellung. Dann gilt
\begin{enumerate}
    \item Die Matrixkoeffizienten von $ \tau$ sind stetig.
    \item Die Matrixkoeffizienten von $ \tau$ sind 
        $\LL^2(K)$-Funktionen.
\end{enumerate}
\end{lemma}
\begin{proof}
Da $K$ kompakt ist, folgt die zweite Aussage direkt aus 
der ersten. Zu zeigen ist also nur die Stetigkeit.\\
Da $ \tau$ eine Darstellung ist, ist die Abbildung
$$ K\times V\to V, (k, v)\mapsto \tau(k)v$$
stetig. Insbesondere ist für festes $ v \in V$ also auch 
die Abbildung
$$ K\to V, k \mapsto \tau(k)v$$
stetig. Da Skalarprodukte auf Hilberträumen stets stetig
sind, folgt, dass die Abbildung 
$$ \varphi\colon K\to \C, k \mapsto \langle \tau(k)v, w 
\rangle $$
stetig, da eine Komposition stetiger Abbildungen ist.
\end{proof}
Im Folgenden wollen wir zeigen, dass es sich bei den
Matrixkoeffizienten um eine Orthonormalbasis des Raums
$\LL^2(K)$ handelt. \\
Dazu wählen wir uns zuerst einen Repräsentanten 
$( \tau, V_ { \tau })$ jeder Äquivalenzklasse aus 
$\UDualFin{K}$ sowie eine Orthonormalbasis $e_1, \dots, e_n$
jedes $ V_ \tau$.\\
Dann führen wir folgende Notation ein, um die nachfolgenden
Sätze ein wenig angenehmer aufschreiben zu können:
$$ \tau_{ij}\colon K \to \C, k\mapsto \langle \tau(k)e_i, e_j 
\rangle.$$
Also meinen wir damit den Matrikoeffizienten von $k$
an der Stelle $(i, j)$. 
\begin{satz}\label{thm:orthoPropsMatrixCoeffs}
Mit obiger Notation gelten folgende Aussagen:
\begin{enumerate}
    \item Für $ \tau \neq \rho$ gilt für alle $1 
    \leq i,j \leq \dim(V_ \tau), 1 \leq r,s \leq 
    \dim(V_ \rho)$:
        $$\langle \tau_{ij}, \rho_{rs} \rangle_{\LL^2} =0.$$
    \item Für $ \tau$ gilt
        $$ \langle \tau_{ij}, \tau_{rs} \rangle = 
        \frac{1}{\dim(V_ \tau)}\delta_{ir} \delta_{js}.$$
\end{enumerate}
\end{satz}
Bevor wir den Satz beweisen, ziehen wir ein unmittelbares
Korollar.
\begin{korollar}\label{cor:matrixCoeffsOrthoSys}
    $ ( \dim(V_ \tau) \tau_{ij})_{ \tau, i, j}$ bildet ein 
    Orthonormalsystem in $ \LL^2(K)$.
\end{korollar}
Nun beweisen wir den obigen Satz 
\ref{thm:orthoPropsMatrixCoeffs}:
\begin{proof}
\begin{enumerate}
    \item Seien $ \tau \neq \rho $ Repräsentanten aus 
        $\UDualFin{K}$.\\
        Dann betrachte eine lineare Abbildung $ T\colon 
        V_ \tau \to V_ \rho$ und definiere dafür die 
        Abbildung $S_T\colon V_ \tau\to V_ \rho$ durch 
        $$S_T = \int_{K} \rho( k^{-1}) T \tau(k) \,
        \diff k.$$
        Wir bezeichnen $S_T$ ab jetzt nur noch mit $S$.
        Dann gilt die folgende Gleichheit: 
        $S \tau(k) = \gamma(k) S$, also ist $S$ ein 
        Vertauscher der Darstellungen $ \tau$ und $ \rho$.
        Dazu sehen wir folgende Rechnung: 
	\begin{align*}
		\rho(k) S 
		&= \rho(k)\left(\int_{K}\rho( m^{-1}) 
			T \tau(m)\,\diff m \right)\\
		&= \int_{K} \rho(k)\rho( m^{-1}) T  
		        \tau(m)\,\diff m  \\
		&= \int_{K} \rho(k m^{-1}) T \tau(m) 
		\,\diff m\\
		&= \int_{K} \rho( (m k^{-1})^{-1}) 
		T \tau(m k k^{-1})\,\diff m \\
		&= \int_{K} \rho(m k^{-1})^* T 
		\tau(m k k^{-1}) \,\diff m  \\
	\end{align*}
	Nun können wir den Satz \ref{thm:propsModularFunction}
	verwenden und folgern, dass
	\begin{align*}
		\rho(k)S 
		&= \int_{K} \rho(m)^* T \tau(mk) 
		\,\diff m\\
	        &= \int_{K} \rho( m^{-1}) T \tau(m) \tau(k) 
		\,\diff m \\
		&= \left(\int_{K} \rho( m^{-1}) T \tau(m) 
		\,\diff m \right)( \tau(k))\\
		&= S \tau(k)
	\end{align*}
	Darauf können wir nun das Lemma von Schur bzw. das 
	Korollar \ref{cor:HomSpaceOneDimensional} verwenden.
	Denn da $ \tau \neq \rho$ gilt, sind die beiden
	Darstellungen nicht unitär äquivalent. Oben haben
	wir jedoch nachgerechnet, dass die Abbildung $S$
	ein Vertauscher zwischen den beiden Darstellungen
	ist. Somit muss $S=0$ gelten.\\
	Nun wollen wir die Eigenschaften solcher $S_T$ 
	ausnutzen. Dafür wählen wir zuerst zwei 
	Orthonormalbasen $(e_j)$ und $(f_s)$ von $V_ \tau$
	bzw. $ V_ \rho$. und betrachten die lineare
	Abbildung $ T_ { js } $, welche definiert wird
	durch $T_{ js}(v) = \langle v, e_j \rangle f_s$.
	Für diese Abbildung wählen wir jetzt $S_ { js } 
	\coloneq S_ { T_ { js }   }$. Dann erhalten
	wir unmittelbar unter Ausnutzung der obigen 
	Rechnung
	\begin{align*}
		0 
		&= \langle S_{js}e_i, f_r \rangle \\
		&= \int_{K} \langle \rho( k^{-1})
		T_{js} \tau(k)e_i, f_r \rangle  \,\diff k\\
		&= \int_{K} \langle \rho( k^{-1})
		\langle \tau(k)e_i, e_j \rangle f_s, f_r 
		\rangle \,\diff k \\
		&= \int_{K} \langle \tau(k)e_i, e_j \rangle  
		\langle \rho( k^{-1})f_s, f_r \rangle 
		\,\diff k\\
		&= \int_{K} \langle \tau(k)e_i, e_j \rangle
		\langle f_s, \rho(k) f_r \rangle \,\diff k\\
		&= \int_{K} \langle \tau(k)e_i, e_j \rangle 
		\overline{\langle \rho(k)f_r, f_s  \rangle} 
		\,\diff k\\
		&= \int_{K} \tau_{ij}(k) 
		\overline{\rho(k)_{rs}}  \,\diff k \\
		&= \langle \tau_{ij}, \rho_{rs} 
		\rangle_{\LL^2}.
	\end{align*}
	Dies zeigt den ersten Teil der Aussage.
    \item Nun betrachten wir den Fall, dass wir den
	    gleichen Repräsentanten der Darstellungen
	    verwenden, also $ \langle \tau_{ij}, \tau_{rs} 
	    \rangle_{\LL^2}$. Obige Rechnung liefert 
	    erneut $$ \langle S_{js}e_i, e_r \rangle 
	    = \langle \tau_{ij}, \tau_{rs}\rangle_{\LL^2}.$$
	    Nun ist jedoch $S \in \Hom_{G}(V_\tau, V_\tau)$.
	    Damit folgt aus dem \hyperref[lem:Schur]
	    {Lemma von Schur}, dass $S_{js} = \lambda 
	    \id_{ V_ \tau}$. Somit muss für $i \neq r$
	    folgen, dass $\langle S_{js} e_i, 
	    e_r\rangle=0$.\\
	    Wir wollen nun noch zeigen, dass auch für
	    $j \neq s$ die Gleichheit mit $0$ folgt. Seien
	    also $j \neq s$. Wir zeigen dann sogar,
	    dass $S_{js} = 0$. Dazu betrachten wir ein
	    typisches Argument in der Theorie irreduzibler 
	    Darstellungen. Es gilt:
	    \begin{align*}
		   \tr( S_{js}) 
           &= \tr\left( \int_{K} \tau( k^{-1}) T_{js} 
           \tau(k) \,\diff  k\right)\\
           &= \int_{K}  \tr( \tau(k)^{-1} T_{js} \tau(k))
           \,\diff  k\\
           &= \int_{K} \tr( \tau(k) \tau(k)^{-1} T_{js})
           \,\diff  k\\
           &= \int_{K} \tr( T_{js}) \,\diff k
	    \end{align*}
	    Wenn wir nun jedoch die Darstellungsmatrix von
        $T_{js}$ bezüglich der gewählten Orthonormalbasis
        $(e_i)$ betrachten, dann fällt auf, dass $j\neq s$
        impliziert, dass auf der Diagonale nur $0$ steht.
        Somit ist $ \tr( T_{js}) = 0$ und somit auch
        $ \tr(S_{js}) = 0$. Gleichzeitig folgt aus 
        $S_{js} = \lambda \id$ auch, dass 
        $ \tr(S_{js}) = \lambda^{\dim(V_ \tau)}$. Somit
        muss $ \lambda = 0 $ gelten und damit ist 
        $S_{js} = 0$. Dies zeigt wieder, dass 
        $ \langle \tau_{ij}, \tau_{rs} \rangle_{\LL^2}=0$
        ist, wenn $i \neq r$ oder $ j \neq s$.\\
        Was noch zu zeigen bleibt, ist der Fall $i = r$
        und $j=s$. Wir erhalten $S_{jj} = \lambda_j \id$
        für alle $j$. Somit folgt aus obiger Rechnung, dass
        $$ \lambda_j = \langle \lambda_j e_i, e_i \rangle  
        = \langle S_{jj}e_i, e_i \rangle = \langle \tau
        _{ij}, \tau_{ij} \rangle_{\LL^2}$$
        und somit $ \langle \tau_{ij}, \tau_{ij} 
        \rangle_{\LL^2} $ nicht von $i$ abhängt. Ziel ist
        es jetzt, zu zeigen, dass alle $ \lambda_j$ gleich
        sind. Da$ \tau_{ij}$ eine unitäre Darstellung ist, 
        folgt $ \tau_{ij}(k)=\overline{\tau_{ji}( k^{-1})}$.
        Somit gilt:
        \begin{align*}
            \lambda_j 
            &= \langle \tau_{ij},\tau_{ij}\rangle_{\LL^2}\\
            &= \int_{K} \tau_{ij}(k)\overline{\tau_{ij}(k)}
            \,\diff k \\
            &= \int_{K} \overline{\tau_{ji}(k^{-1})} 
            \tau_{ji}( k^{-1})\,\diff k \\
            &= \int_{K} \overline{ \tau_{ji}( k)}
            \tau_{ji}( k)\,\diff k\\ %TODO: den Pushforward
            % der Inversionsabbildung aufschreiben und hier
            % sauber referenzieren
            &= \langle \tau_{ji},\tau_{ji} \rangle_{\LL^2}\\
            &= \lambda_i
        \end{align*}
        wobei im vorletzten Schritt Satz ?? %TODO: hier referenz einbauen
        verwendet wurde. Somit muss es sich stets um den
        gleichen Wert bei allen $ \lambda_j$ handeln.
        Diesen nennen wir $ \lambda$. Wir müssen jetzt 
        nur noch zeigen, dass $ \lambda = 
        \frac{1}{\dim(V_ \tau)}$ gilt. Dazu schreiben
        wir $ n\coloneq \dim(V_ \tau)$. Durch Prüfen auf
        unserer gewählten Basis $(e_i)$ sehen wir schnell
        ein, dass $ \sum_{i} T_{ii} = \id $. Somit folgt
        \begin{align*}
            (n \lambda)\id 
            &= \sum_{i} S_{ii}\\
            &= \sum_{i} \int_{K} \tau( k^{-1}) T_{ii} 
             \tau(k)\,\diff k\\
            &= \int_{K} \tau( k^{-1}) \sum_{i}T_{ii} \tau(k)
            \,\diff  k\\
            &= \int_{K} \tau( k^{-1}) \id\tau(k) \,\diff k\\
            &= \id
        \end{align*}
        Somit muss $ \lambda = \frac{1}{n}$ folgen, was die
        zu zeigende Aussage beweist.
\end{enumerate} 
\end{proof}
Damit wissen wir schon einmal, dass die Matrixkoeffizienten
ein Orthonormalsystem bilden. Der Frage, ob sie auch eine
Orthonormalbasis bilden, werden wir im nächsten Abschnitt 
nachgehen, wenn wir den Satz von Peter-Weyl beweisen werden.
\section{Vollständigkeitseigenschaften}
In diesem Abschnitt werden wir mehrere Sätze beweisen,
die unter dem Satz von Peter-Weyl bekannt sind. Dazu benötigen
wir jedoch noch ein wenig Vorarbeit:
\begin{definition}[Dirac-Funktionen, Dirac-Netz]
Auf einer lokalkompakten, topologischen Gruppe $G$ heißt eine
Funktion $ \phi \in C_c(G)$ eine \underline{Dirac-Funktion}
\index{Dirac-Funktion}, falls
\begin{itemize}
	\item $ \phi \geq 0$,
	\item $ \int_{G} \phi(x) \,\diff x =1  $,
	\item $ \phi( x^{-1}) = \phi(x)$.
\end{itemize}
Sei $\mathcal U$ die Menge aller offenen Umgebungen der $1_G$.
Dann heißt eine Familie von Dirac-Funktionen $( \phi_U)_ 
{ U \in \mathcal U  } $ ein \underline{Dirac-Netz}
\index{Dirac-Netz}, falls $ \supp{ \phi_U} \subseteq U$
für alle $U \in \mathcal{U}$ gilt.
\end{definition}
% TODO: Faltung und Faltung glättet L²-Funktionen zu C-
% Funktionen
\begin{satz}[Peter-Weyl I]\label{thm:peterWeyl}
    Sei $K$ eine kompakte, topologische Gruppe sowie 
    $ \tau$ Repräsentanten der Äquivalenzklassen aus 
    $\UDualFin{K}$ mit dem zugehörigen Orthonormalsystem
    $( \tau_{ij})_{ \tau, i, j}$ aus 
    \ref{thm:orthoPropsMatrixCoeffs}. Dann gilt
    \begin{enumerate}
        \item dieses Orthonormalsystem ist sogar eine
            Orthonormalbasis.
        \item Die linksreguläre Darstellung $(L, \LL^2(K))$
            aus Beispiel \ref{ex:leftRegularRepr}
            zerfällt in eine direkte Summe 
            endlichdimensionaler, irreduzibler 
            Darstellungen.
    \end{enumerate} 
\end{satz}
\begin{proof}
\begin{enumerate}
	\item Sei $ \tau \in \UDualFin{K}$ und $ M_ \tau$ der
	Unterraum, der durch die Matrixkoeffizienten $ M_ 
	\tau$ aufgespannt wird. Zuerst sehen wir schnell ein,
	dass für einen Matrixkoeffizienten $ h(k) = 
	\langle \tau(k)v, w  \rangle $ gilt
	\begin{align*}
		h^*(k) &= \overline{h( k^{-1})} = \langle 
		\tau(k)w,v \rangle \in M_ \tau\\
		L_{k_0}h(k) &= h( k_0^{-1}k) = \langle 
		\tau(k)v, \tau(k_0)w \rangle \in M_ \tau\\
		R_{k_0}h(k) &= h( k k_0) = \langle \tau(k) 
		\tau(k_0)v, w \rangle in M_ \tau.
	\end{align*}
	Somit ist $ M_ \tau$ abgeschlossen unter Links- und
	Rechtstranslationen sowie Konjugation. Nun 
	bezeichnen wir mit $M$ den $\LL^2(K)$-Abschluss 
	aller $M_ \tau$. Wir wollen zeigen, dass $M = 
	\LL^2(K)$ oder äquivalent, dass $ M^\perp = \{ 0 \}$.
	Angenommen, $M^\perp \neq \{ 0 \}$. 
	\begin{claim}
		$\orthoComp{M}$ 
		enthält eine stetige Funktion ungleich	der
		Nullfunktion.
	\end{claim}
	Sei $ H \neq 0$ in $M^\perp$ und wähle ein Dirac-Netz
	$ (\phi_U)_U$. Dann konvergiert $ \phi_U \ast H$
	in der $\LL^2$-Norm gegen $H$. Da $\orthoComp{M}$ 
	abgeschlossen unter Translationen ist, und $ \phi_U
	\in C_c(K)$ für alle $U \in \mathcal{U}$ folgt 
	direkt, dass $ \phi_U \ast H \in \orthoComp{M}$.
	Da die Faltung Funktionen glättet und es ein $U \in
	\mathcal{U}$ geben muss mit $ \phi_U \ast H \neq 0$
	(sonst wäre $H$ durch die $\LL^2$-Konvergenz bereits
	die Nullfunktion), folgt die Behauptung.\\
	Wir wählen also eine stetige Funktion $F_1 \in 
	\orthoComp{M} $ mit $ F_1 \neq 0$. Durch Translation
	und Reskalierung können wir annehmen, dass $F_1(e)
	>0$. \\
    Wir betrachten nun eine neue Funktion 
    $F_2(x) = \int_{K} F_1( y^{-1}xy)  \,\diff y $. 
    \begin{claim}
      Dann hat $F_2$ folgende Eigenschaften:
      \begin{enumerate}
          \item $F_2 \in \orthoComp{M}$,
          \item $F_2$ ist invariant unter Konjugation,
          \item $F_2(e)>0$.
      \end{enumerate}
    \end{claim}
    \begin{enumerate}
        \item Sei $ \tau_{vw}$ ein Matrixkoeffizient. Dann
            ist 
            \begin{align*}
                \langle F_2, \tau_{vw} \rangle_{\LL^2}
                &= \int_{K} F_2(x) \overline{\tau_{vw}(x)}
                \,\diff  \\
                &= \int_{K} \int_{K} F_1( y^{-1}xy)   
                \,\diff y\,\overline{\tau_{vw}(x)}\,
                \diff x\\
                &= \int_{K} \int_{K} F_1( y^{-1}xy) 
                \overline{\tau_{vw}(x)}\,\diff y\,\diff x\\
                &= \int_{K} \int_{K} F_1( y^{-1}xy) 
                \overline{\tau_{vw}(x)}\,\diff x\,\diff y\\
                &= \int_{K} \int_{K} F_1(x) \overline{ 
                \tau_{vw}( y x y^{-1})}\,\diff x\,\diff y\\
                &= \int_{K} \int_{K} F_1(x) 
                \overline{L_{y}(R_{ y^{-1}} \tau_{vw})(x)}
                \,\diff x  \,\diff y\\
                &= \int_{K} \langle F_1, L_y( R_ { y^{-1} }
                \tau_{vw})\rangle \,\diff y
            \end{align*}
            Nun haben wir bereits gesehen, dass die
            Matrixkoeffizienten abgeschlossen unter 
            Links- und Rechtsverschiebungen sind. Damit
            ergibt sich, dass $L_y(R_ { y^{-1} }\tau_{vw})$
            stets ein Matrixkoeffizient und damit
            insbesondere in $M$ ist. Somit erhalten
            wir
            $$ \int_{K} \langle F_1,
            L_y(R_ { y^{-1} } \tau_{vw} )\rangle \,\diff y
            = \int_{K} 0 \,\diff y =0.$$
            Damit ist $F_2 \in \orthoComp{M}$.
        \item Wir wollen zeigen, dass für alle
            $x,z \in K$ gilt, dass $F(x) = F( z x z^{-1})$.
            Dies folgt unmittelbar aus der Links- und
            Rechtsinvarianz des Haarmaßes.
        \item Es ist 
            $$F_2(e) = \int_{K} F_1( y^{-1} e y) 
            \,\diff y = \int_{K} F_1( e) \,\diff > 0. $$
    \end{enumerate}
    Damit folgt, dass $F_2$ die behaupteten Eigenschaften
    hat. Nun betrachten wir die Funktion $F(x) = F_2(x) + 
    \overline{F_2( x^{-1})}$. Diese Funktion ist 
    offensichtlich stetig. Zudem rechnet man schnell nach,
    dass $F \in \orthoComp{M}, F(e)>0$ und $F= F^*$.\\
    Nun betrachten wir $T\colon\LL^2(K)\to \LL^2(K), f\mapsto 
    f\ast F$. Man rechnet schnell unter Ausnutzen des
    Satzes von Fubini nach, dass $T$ selbstadjungiert
    ist, da $F(x) = \overline{F( x^{-1})}$ gilt. Zudem 
    gilt 
    $$ T(f)(x) = \int_{K} f(y) F( y^{-1}x) \,\diff y.$$
    Also handelt es sich bei $T$ um einen Integraloperator
    mit stetigem Kern $ (x,y)\mapsto F( y^{-1}x)$. 
    %TODO: Integraloperatoren sind Hilbert-Schmidt-Operatoren
    %TODO: Hilbert-Schmidt-Operatoren sind kompakt
    Aus ??? folgt, dass es sich bei $T = T^* \neq 0$ (nicht 0,
    da $F$ stetig ist und an einer Stelle nicht verschwindet)
    um einen Hilbert-Schmidt-Operator, insbesondere also um 
    einen kompakten Operator handelt.\\
    %TODO: Spektralsatz für kompakte Operatoren + Referenz
    Aus dem Spektralsatz für kompakte, selbstadjungierte
    Operatoren folgt die Existenz eines Eigenwerts 
    $ \lambda\neq 0$, für welchen der zugehörige Eigenraum
    $V_ \lambda$. Dieser Eigenraum ist abgeschlossen unter
    Linksverschiebung $L_k$ für alle $k \in K$, denn es
    gilt für $ f \in V_ \lambda$:
    \begin{align*}
	    ((L_k f) \ast F)(x) 
	    &= \int_{K} f( k^{-1}y)F( y^{-1}x) \,\diff y\\
	    &= \int_{K} f(y) F(k y^{-1} x) \,\diff y \\
	    &= \int_{K} f(y x^{-1})F(k y^{-1}) \,\diff y\\
	    &= \int_{K} f(y x^{-1} k) F(k y^{-1} k^{-1}) 
	    \,\diff y\\
	    &= \int_{K} f(y x^{-1}k) F( y^{-1}) \,\diff y\\
	    &= \int_{K} f(y) x^{-1}) F( y^{-1} k^{-1}) 
	    \,\diff y\\
	    &= \int_{K} f(y) F( y^{-1} k^{-1}x) \,\diff y\\
	    &= (f\ast F)( k^{-1}x)\\
	    &= L_k(f\ast F)( k^{-1}x)\\
	    &= L_K( \lambda f)( k^{-1}x)\\
	    &= \lambda L_k f ( k^{-1}x)
    \end{align*}
    Somit gilt $ (L_K f) \ast F = \lambda L_k f$ und damit
    ist der Eigenraum abgeschlossen unter Linksverschiebung.
    In der Rechnung wurde im vierten Schritt verwendet, dass
    $F$ konjugationsinvariant ist, weil $F_2$ 
    konjugationsinvariant ist.\\
    Demnach können wir auf $ V_ \lambda$ mit der Linkswirkung
    eine unitäre Darstellung von $K$ finden. Da $V_ \lambda$
    endlichdimensional ist, finden wir mit Lemma 
    \ref{lem:FiniteDimReprSumOfIrredRepr} eine irreduzible
    Unterdarstellung $W \subseteq V_ \lambda \subseteq 
    \orthoComp{M}$. Für $f,g \in W$ definieren wir
    $h(k) = \langle L_k f, g \rangle $ den zugehörigen
    Matrixkoeffizienten der obigen Darstellung. Dann gilt
    nach Definition
    $$h(k) = \int_{K} f( k^{-1}x)\overline{g(x)}\,\diff x.$$
    Gleichzeitig gilt $h = \overline{g \ast f^*} \in 
    \orthoComp{M}$. Da es sich bei $h$ jedoch um einen 
    Matrixkoeffizienten handelt, folgt auch $h \in M$. Somit
    $ \langle h,h \rangle =0$. Damit muss $L_k$ auf $V_ \lambda$
    die Nullfunktion sein, was einen Widerspruch darstellt.
    Somit muss $M = \LL^2(K)$ gelten.
    \item In obigem Beweis haben wir gezeigt, dass
	    $$\LL^2(K) = \overline{ 
	    \bigotimes_{ \tau \in \UDualFin{K}} M_ \tau},$$
	    wobei alle $M_ \tau$ endlichdimensional sind
	    (genauer gesagt Dimension $\dim( \tau)^2$ hat).
	    Da alle $ M_ \tau$ abgeschlossen unter Links- und
	    Rechtsverschiebung sind, zerfällt die linksreguläre
	    Darstellung $(\LL^2(K), L)$ in eine direkte Summe
	    endlichdimensionaler Darstellungen. Aus Lemma
	    \ref{lem:FiniteDimReprSumOfIrredRepr} folgt
	    die zu zeigende Aussage.
    
\end{enumerate}

\end{proof}
\section{Anwendungen}
Mit dem \hyperref[thm:peterWeyl]{Satz von Peter-Weyl}
lassen sich einige sehr mächtige Aussagen folgern.
Dabei spielen die treuen Darstellungen eine besondere
Rolle:
\begin{definition}[treue Darstellung]
    Eine Darstellung auf einer topologischen Gruppe $G$
    $ \rho\colon G\to \GLS{V}$ und einem Vektorraum $V$
    heißt \underline{treu}\index{treue Darstellung},
    falls injektiv ist.
\end{definition}
\begin{bemerkung}
Eine Darstellung $ \rho$ ist also genau dann treu, wenn
$\ker( \rho) = \{ e_G \} $ ist. Das bedeutet also, dass
nur das triviale Element der "identischen Wirkung" 
entspricht.
\end{bemerkung}
Zuerst wollen wir die obige Dichtheit in $\LL^2(G)$
auf die Dichtheit in $C(K)$ erweitern, wobei man hier
beachten muss, dass bezüglich unterschiedlicher Normen
gearbeitet wird. Um die Dichtheit zu folgern, erinnern
wir uns zuerst an den Satz von Stone-Weierstraß in der
Formulierung aus \cite{reprCompLieGroups}:
\begin{satz}[Stone-Weierstraß]
	\label{thm:stoneWeierstrass}
	Sei $ X$ ein kompakter Hausdorff-Raum. Sei
	$ \mathcal{A} \subseteq C(X, \C)$ eine
	Unteralgebra von $C(X, \C)$, welche
	\begin{enumerate}
		\item alle konstanten Funktionen
			enthält.
		\item Punkte trennt, es also für 
			$x_1 \neq x_2 \in X$ eine
			Funktion $f \in 
			\mathcal{A}$ gibt, sodass
			$f(x_1)\neq f(x_2)$ und
		\item abgeschlossen unter komplexer
			Konjugation ist.
	\end{enumerate}
	Dann ist $ \mathcal{A}$ dicht in $C(X, \C)$. 
\end{satz}
Damit können wir nun die Dichtheit der 
Matrixkoeffizienten in $C(X, \C)$ folgern, wobei wir
uns vorher noch Gedanken um das Produkt von 
Darstellungen Gedanken machen müssen:
\begin{definition}[Tensorprodukte von Darstellungen]
Seien $( \rho, V), ( \tau, W)$ endlichdimensionale
Darstellungen einer kompakten Gruppe $K$. Dann
erhalten wir mittels $\rho\otimes \tau\colon K\to 
V\times W$ gegeben durch
$$ (\rho \otimes \tau)(k)(v, w) = \rho(k)(v) \otimes 
\tau(k)(w)$$
eine neue Darstellung. Diese Operation nennen wir
das \underline{Tensorprodukt der Darstellungen}
\index{Tensorprodukt von Darstellungen} $ \rho$ und
$ \tau$.
\end{definition}
Dass diese Darstellung wohldefiniert ist, rechnet man
schnell nach. Spannender hingegen ist zu prüfen, ob
es sich bei dieser Darstellung um eine unitäre 
Darstellung handelt.
\begin{lemma}
Seien $( \rho, V)$ und $( \tau, W)$ Darstellungen 
einer kompakten Gruppe $ K$. Auf dem Tensorprodukt
$V \otimes W$ sei das kanonische Skalarprodukt
$$ \langle v_1\otimes w_1, v_2 \otimes w_2 
\rangle = \langle v_1, v_2 \rangle \cdot 
\langle w_1, w_2 \rangle $$
gegeben. Dann ist $ \rho \otimes \tau$ unitär
bezüglich dieses Skalarproduktes.
\end{lemma}
\begin{proof}
Für $v_1\otimes w_1, v_2 \otimes w_2 \in V\otimes W$
und $k \in K$ gilt:
\begin{align*}
\langle (\rho \otimes \tau)(k)(v_1 \otimes w_1),
( \rho \otimes \tau)(k)( v_2 \otimes w_w)\rangle
&= \langle \rho(k)v_1 \otimes \tau(k)w_1,
\rho(k)v_2 \otimes \tau(k) w_2\rangle \\
&= \langle \rho(k)v_1, \rho(k)v_2 \rangle
\cdot \langle \tau(k)w_1, \tau(k)w_2 \rangle \\
&= \langle v_1, v_2 \rangle \cdot 
\langle w_1, w_2 \rangle, 
\end{align*}
wobei wir im letzten Schritt benutzen, dass die 
Darstellungen unitär bezüglich ihrer jeweiligen
Skalarprodukte sind.
\end{proof}
\begin{korollar}\label{cor:matrixCoeffsDenseInC}
Das Orthonormalsystem Matrixkoeffizienten 
$ ( \tau_{ij})_{ \tau, i, j}$ hat dichten Spann in 
$C(X, \C)$.
\end{korollar}
\begin{proof}
Nach \hyperref[thm:stoneWeierstrass]{Satz von 
Stone-Weierstraß} müssen wir nur
zeigen, dass es sich beim Spann Matrixkoeffizienten 
um eine Algebra handelt,
welche den obigen Anforderungen genügt. Zuerst 
rechnen wir nach, dass es sich tatsächlich um eine 
Algebra handelt. Für zwei Darstellungen$ (\tau, V)$ 
und $(\rho, W)$ rechnen wir für ihre 
Matrixkoeffizienten$ \tau_{vw}$ und $ \rho_{xy}$ 
bezüglich Vektoren $v,w \in V$ und $x,y \in W$ 
nach, dass $ \tau_{vw} \cdot \rho_{xy} =
( \tau \otimes \rho)_{v\otimes x, w\otimes y}$
gilt. Dazu sehen wir für alle $ k \in K$: 
\begin{align*}
(\tau_{vw} \otimes \rho_{xy})(k)
&= \langle \tau(k)v,w \rangle
\langle \rho(k)x,y \rangle \\
&= \langle \tau(k)v\otimes \rho(k)x, w \otimes y 
\rangle \\
&= \langle ( \tau \otimes \rho)(k) v \otimes x,
w \otimes y
\rangle. 
\end{align*}
Dies zeigt die Aussage und die Wohldefiniertheit
der Multiplikation. Somit handelt es sich
tatsächlich um eine Algebra. Wir müssen nun noch
die übrigen Anforderungen prüfen:
\begin{enumerate}
\item Wir zeigen, dass die konstante 
	Einsfunktion enthalten ist. Der Rest
	folgt aus Skalarmultiplikation. Dazu wählen
	wir als Darstellung den Vektorraum $\C$
	und die triviale Darstellung, also
	$$ \rho\colon K\to \GLS{\C}, k\mapsto \id_\C.$$
	Da $\C$ eindimensional ist, erhalten wir
	die Irreduzibilität sofort. Des Weiteren
	gilt
	$$ \langle \rho(k)1, 1 \rangle 
	= \langle 1, 1 \rangle = 1 $$
	für alle $ k \in K$. Somit ist die konstante
	Einsfunktion enthalten.
\item Dass die Matrixkoeffizienten abgeschlossen
	unter komplexer Konjugation sind, folgt 
	direkt aus folgender Rechnung:
	$$ \overline{\langle  \tau(k)v, w \rangle} = 
	\overline{\langle v, \tau(k)^{*}w  \rangle} =
	\langle \tau( k^{-1})w, v \rangle,$$
	was auch einem Matrixkoeffizienten entspricht.
	Damit ist die Abgeschlossenheit unter 
	komplexer Konjugation gezeigt.
\item Die eigentlich spannende Eigenschaft ist, dass
    die Matrixkoeffizienten Punkte trennen\footnote{
    Hier wird der Beweis aus der ersten Antwort dieses 
    StackExchange-Beitrags ausformuliert: 
\href{https://math.stackexchange.com/questions/1479222/peter-weyl-theorem-versions}{\textcolor{blue}{Peter-Weyl theorem versions}}.}. 
    Zuerst überlegen wir uns, dass es genügt zu zeigen, 
    dass 
    es im Fall von topologischen Gruppen $K$ genügt, 
    dass es für alle $x \in K$ eine stetige 
    Funktion $f\colonK\to \C$, sodass $f(x)\neq f(1)$.\\
    Um dies zu sehen, seien $x_1 \neq x_2$. Dann gibt es
    nach Annahme eine stetige Funktion $f\colon K\to \C$,
    sodass $f( x_1^{-1} x_2) \neq f(1) $.  Dann ist
    insbesondere die Abbildung $ L_ { x_1 } f $ als 
    Verkettung stetiger Abbildungen stetig und es gilt:
    $$ L_ { x_1 } f(x_2) = f( x_1^{-1} x_2) \neq f(1)
    = f( x_1^{-1} x_1) = L_{x_1} f(x_1).$$
    Somit ist die gewünschte Aussage gezeigt.\\
    Wir wollen also zeigen, dass es für alle $g\in K$ einen
    Matrixkoeffizienten $ \tau_{vw}$ gibt, sodass
    $ \tau_{vw}(g) \neq \tau_{vw}(1)$.
    Dazu werden wir eine stärkere Eigenschaft zeigen: für
    alle $g \in G$ gibt es eine endlichdimensionale, 
    unitäre, irreduzible Darstellung $ \rho$, sodass
    $ \rho(g)\neq \rho(1)$. Daraus folgt dann direkt, dass
    es einen unterschiedlichen Matrixkoeffizienten geben
    muss, sonst wären die Abbildungen gleich.\\
    Dazu betrachten wir den Schnitt $H$ der Kerne aller
    endlichdimensionalen, irreduziblen, unitären 
    Darstellungen. Da Kerne normal sind und der Schnitt
    normaler Untergruppen normal ist, ist $H \unlhd K$.
    Ebenso sehen wir ein, dass $H$ abgeschlossen und
    damit kompakt ist.\\
    Damit erhalten wir auf $K/H$ eine wohldefinierte
    Gruppenstruktur. Zudem ist $K/H$ eine kompakte Gruppe.
    Somit erhalten wir auf $K, H$ und $K/H$ jeweils
    ein Haarmaß und können die folgende Formel verwenden
    (siehe z.B. \cite[Theorem 1.5.3]{principlesHarmAna}):
    $$ \int_{K} f(x)  \,\diff x = \int_{K/H} \int_{H} 
    f(xh)  \,\diff h  \,\diff x$$
    für jedes $ f \in \LL^1(K)$.\\
    Damit können wir folgern, dass die Quotientenabbildung
    $q^*\colon \LL^1(G/H) \to \LL^1(G), f\mapsto f\circ q$ 
    eine Isometrie ist. Denn für $ f \in \LL^1(G/H)$
    gilt mit obiger Formel:
    \begin{align*}
        \int_{K} (f\circ q)(x)  \,\diff x
        &= \int_{K/H} \int_{H} (f\circ q)(xh)
        \,\diff h  \,\diff x \\
        &= \int_{K/H} \int_{H} f(xhH) \,\diff h\,\diff x\\
        &= \int_{K/H} \int_{H} f(xH) \,\diff h \,\diff x\\
        &= \int_{K/H} f(xH)  \,\diff x,
    \end{align*}
    wobei im letzten Schritt verwendet wurde, dass 
    $\diff h$ ein normiertes Haarmaß auf $H$ ist. 
    Insbesondere schränkt sich $q^*$ also zu einer
    Isometrie zwischen $\LL^2(K/H)$ und $\LL^2(K)$
    ein. Sei nun $ \tau_{vw}$ ein Matrixkoeffizient.
    Dann ist $ H \subseteq\ker( \tau_{vw})$ nach der
    Konstruktion von $ \tau$. Somit erhalten wir durch
    die universelle Eigenschaft der Faktorgruppe 
    \cite[Satz 2.6]{algebra} eine Abbildung $ \overline{ 
    \tau_{vw}}$, sodass $ \overline{ \tau_{vw}} \circ q 
    = \tau_{vw}$ gilt. Damit sind alle Matrixkoeffizienten
    im Bild von $\LL^2(K/H)$.\\
    Da Bilder abgeschlossener Menge unter Isometrien 
    zwischen vollständigen Räumen wieder eine abgeschlossen
    ist, folgt, dass $q^*(\LL^2(K/H))$ abgeschlossen in
    $\LL^2(K)$ ist. Da auch alle Matrixkoeffizienten 
    enthalten sind, muss $\im(q^*) = \LL^2(K)$ gelten.
    Falls $H\neq \{ 1 \} $, folgt aus dem Lemma von 
    Urysohn, dass es eine nicht-konstante Funktion $f$
    auf $ H$ in $ \LL^2(K)$ gibt. Da $\im(q^*) = \LL^2(K)$,
    folgt, dass $f \in \im(q^*)$. Jedoch gilt für eine
    Funktion $ g \in \im(q^*)$, dass $g$ konstant auf $H$
    ist, denn für $ (g'\circ q)(h) = g'(H) = (g'\circ q)
    (h')$ mit $h, h' \in H$, wenn $ g' \circ q = g$. Dies
    ist ein Widerspruch und somit $H= \{ 1 \} $.
    Damit trennen die Matrixkoeffizienten Punkte.
\end{enumerate}
Somit folgt aus dem Satz \ref{thm:stoneWeierstrass}, dass
die Matrixkoeffizienten dicht in $C(K)$ sind.
\end{proof}
Für die Situation von Lie-Gruppen können wir aus dem 
Satz von Peter und Weyl die Existenz treuer Darstellungen
von Lie-Gruppen folgern.
Dann gilt die folgende Eigenschaft:
\begin{satz}\label{thm:compLieGrpsHaveFaithfulRepr}
    Sei $K$ eine kompakte Lie-Gruppe. Dann besitzt $K$
    eine treue Darstellung.
\end{satz}
Bevor wir dies beweisen, müssen wir eine wichtige 
Eigenschaft von Lie-Gruppen festhalten.
\begin{satz}[absteigende Ketteneigenschaft]
    \label{thm:descendingChainProp}
Sei $G$ eine kompakte Lie-Gruppe und $ K_1 \supseteq K_2 
\supseteq K_3 \supseteq \dots$ eine absteigende Folge 
abgeschlossener Untergruppen von $G$. Dann stabilisiert
sich diese Folge, also gibt es ein $ k \in \N$, sodass
$ K_l = K_k$ für alle $l \geq k$.
\end{satz}
Wir wollen den Satz an dieser Stelle nicht beweisen, 
sondern verweisen dafür auf \cite{strucThmCompLieGroups}.
Damit können wir jetzt jedoch Satz 
\ref{thm:compLieGrpsHaveFaithfulRepr} beweisen:
\begin{proof}
Sei $ k \in K \setminus \{ 1\} $. Dann gibt es nach dem
Lemma von Urysohn eine stetige Funktion $f\colon G\to \R$
mit $f(k) \neq f(1)$. Aus dem Korollar 
\ref{cor:matrixCoeffsDenseInC} folgt die Existenz eines
Matrixkoeffizienten $ \tau_{1,vw}$ mit $ \tau_{1,vw}$ mit
einer Darstellung $ \tau_1$ mit $ \tau_{1,vw}(k)\neq 
\tau_{1,vw}(1)$. Dann ist $ K_1 \coloneq\ker( \tau)$ eine 
echte, abgeschlossene Untergruppe von $ K$. Falls 
$ K_1 = \{ 1 \} $, dann ist $ \tau_1$ die gesuchte 
Darstellung.\\
Ansonsten wähle $ g \in K_1\setminus \{ 1\} $ und eine
stetige Funktion $f\colon K\to \R$, sodass $f(k)\neq f(1)$
(wieder mit Urysohn) sowie die zugehörige Darstellung
$ \tau_{2}$ mit $ K_2 \coloneq \ker( \tau_2)$. Dann
ist $ K_1 \cap K_2 \subsetneq K_1$, da $ \tau_2(g)\neq 
\tau_2(1)$. Wenn wir diesen Prozess iterieren, erhalten
wir eine absteigende Folge echt in einander enthaltener, 
abgeschlossener (also insbesondere kompakter) 
Untergruppen von $ K$. Nach \ref{thm:descendingChainProp}
folgt, dass die Schnitte über die Kerne aller Darstellungen
irgendwann trivial werden. Damit folgt, dass die direkte
Summe treu ist.
\end{proof}
Da sich harmonische Analysis zu einem großen Teil damit
beschäftigt, die klassische Fourieranalysis zu 
verallgemeinern, wollen wir am Ende dieser Arbeit wieder
einige Rückschlüsse, für die konkrete Fourieranalysis 
festhalten, die uns der Satz von Peter-Weyl ermöglicht.
\begin{satz}\label{thm:fourierSeriesConverges}
Sei $ f \in \LL^2(\mathbb S^1)$. Dann konvergiert die 
Fourierreihe von $ f$ gegen $ f$ in der $\LL^2$-Topologie.
Darüberhinaus kann jede stetige Funktion $f\colon\mathbb 
S^1\to \C$ gleichmäßig durch trigonometrische Polynome 
angenähert werden.
\end{satz}\label{thm:fourierSeriesConvergesInL2}
Dazu wollen wir uns zuerst kurz an die Definition der
Fourierreihe erinnern. Die Primärquelle dazu ist 
\cite[Abschnitt 4.26]{papaRudin}. 
\begin{definition}[Fourierreihe]
Sei $ f \in \LL^1(\SB^1)$. Dann definieren wir die
\underline{Fourrierkoeffizienten}
\index{Fourrierkoeffizienten} durch
$$\hat{f}\colon\Z\to \C, \hat{f}(n) = 
 \frac{1}{2 \pi}\int_{\SB^1} f(z) z^{-n} \,\diff z.$$
Die \underline{Fourierreihe}\index{Fourierreihe}
von $f$ ist gegeben durch
$$ \sum_{ n = - \infty}^{ \infty} \hat{f}(n) z^{n}.$$
\end{definition}
Bevor wir den Beweis von 
\ref{thm:fourierSeriesConvergesInL2} führen können,
benötigen wir noch den folgenden Satz:
\begin{satz}\label{thm:contGroupHomsMonomials}
Die stetigen Gruppenhomomorphismen $\chi\colon\SB^1\to\SB^1$
sind genau die Funktionen $z\to z^n$ für $n\in \Z$.
\end{satz}
Einen Beweis dieser Aussage findet man zum Beispiel
in \cite[Proposition 7.1.1]{firstCourseHarmAna}.
Nun können wir obigen Satz zeigen:
\begin{proof}
Im letzten Vortrag wurde gezeigt, dass das 
normierte Haarmaß auf $\SB^1$ gegeben ist durch
$ \frac{1}{2 \pi} \diff z$, wobei $\diff z$ die
Einschränkung des Lebesgue-Maßes auf die $\SB^1$
bezeichnet.\\
Da wir alles auf $\SB^1$ betrachten, entspricht
die komplexe Konjugation dem Invertieren von 
Elementen. Damit sehen wir direkt, dass es sich
bei $\hat{f}(n)$ um das $\LL^2$-Skalarprodukt von
$f$ und $z^{n}$ handelt. Wenn wir jetzt zeigen,
dass es sich bei $z\mapsto z^n$ mit $n \in \Z$ um
eine Orthonormalbasis von $\LL^2$ handelt, dann
sehen wir die Aussage sofort ein. Dann handelt es
sich nämlich um die typische Entwicklung eines
Elements eines Hilbertraums in einer ONB (
die Konvergenz dieser wird im Rahmen der 
Funktionalanalysis mit Hilfe der Parsevalschen
Ungleichung gezeigt).\\
Nun wollen wir zeigen, dass es sich bei 
$(z\mapsto z^n)_{n \in \Z}$ tatsächlich um
eine ONB handelt. Dies wollen wir aus dem 
\hyperref[thm:peterWeyl]{Satz von Peter-Weyl}
folgern, indem wir zeigen, dass es sich bei 
diesen Funktionen genau um die Matrixkoeffizienten
handelt.\\
Dazu sei nun im Folgenden $( \rho, V_ \rho)$ eine
irreduzible, unitäre Darstellung. Da $ \SB^1$ abelsch
ist, sehen wir direkt, dass es sich bei allen
$ \rho(x)$ mit $ x \in\SB^1$ um Vertauscher handelt.
Es gilt also $\rho( \SB^1) \subseteq \End_ { \SB^1 } 
(V_ \rho)$. Jedoch ist $ \rho$ irreduzibel und damit 
gilt nach dem \hyperref[lem:Schur]{Lemma von Schur}, 
dass $\rho(\SB^1) \subseteq \C \id_{ V_ \rho}$. Für
alle $ x \in \SB^1$ gibt es also ein $ \alpha(x) 
\in \C$, sodass $ \rho(x) = \alpha(x)\id_ {V_\rho}$.
Da $ \rho$ unitär ist, gilt sogar für alle $x \in 
\SB^1$, dass $ \alpha(x)^{-1} = \rho( x^{-1})1 =
\rho(x)^*1 = \overline{ \alpha(x)}$, wobei
im vorletzten Schritt Lemma 
\ref{lem:unitaryReprMapsInversesToAdjoints}
verwendet wurde. Damit ist
$ \alpha\colon \SB^1 \to \SB^1$ eine wohldefinierte
Abbildung. Wir wollen zunächst zeigen, dass es
sich bei $ \alpha$ um einen stetigen 
Gruppenhomomorphismus handelt. Die Stetigkeit
erhalten wir direkt daraus, dass die Abbildung
$ x \mapsto \rho(x) 1$ stetig ist, da $ \rho$
eine Darstellung ist. Zudem sehen wir, dass
diese Abbildung gleich $ x\mapsto \alpha(x)$ 
ist und somit ist $ \alpha$ stetig. Für die
Gruppenhomomorphismus-Eigenschaft sehen wir,
dass $ \alpha(xy) = \rho(xy)1 = \rho(x) 
\rho(y)1 = \rho(x)( \alpha(y)) = \alpha(x) 
\alpha(y)$ gilt. Somit handelt es sich bei
$ \alpha$ um einen stetigen Gruppenhomomorphismus
von $ \SB^1$ nach $ \SB^1$. Damit können wir den
Satz \ref{thm:contGroupHomsMonomials} verwenden,
um zu folgern, dass $ \alpha(x) = x^n$ für ein
$ n \in \Z$ ist. Damit folgt, dass $ \rho_{11}(x)=
\langle \rho(x)1, 1 \rangle = \alpha(x)$ der
zugehörige Matrixkoeffizient ist.\\
Umgekehrt erhalten wir für jeden stetigen 
Gruppenhomomorphismus $ \alpha\colon\SB^1\to \SB^1$
eine irreduzible, unitäre Darstellung $(\rho_ \alpha, 
\C)$ gegeben durch $ \rho_ \alpha(x)(z) = \alpha(x) z$
für $ z \in \C$ und $ x \in \SB^1$, welche offensichtlich
den Matrixkoeffizienten $ \alpha(x)$ hat.\\
Somit sind alle Matrixkoeffizienten von der Form 
$ z\mapsto z^n$ für $ n \in \Z$ und mit dem
\hyperref{thm:peterWeyl}{Satz von Peter-Weyl} folgt, dass
es sich dabei tatsächlich um eine ONB handelt.\\
Wir wissen aus Korollar \ref{cor:matrixCoeffsDenseInC},
dass die Matrixkoeffizienten dicht in $C(\SB^1)$ sind.
Wie oben gezeigt sind die Matrixkoeffizienten genau durch
die trigonometrischen Polynome $z \mapsto z^n$ gegeben auf
$\SB^1$. Da die Matrixkoeffizienten dichten Spann haben,
folgt, dass sich jedes $ f \in C(\SB^1)$ schreiben lässt
als $f= \sum_{k=- \infty}^{ \infty} c_k z^k$ für gewisse
$ c_k \in \C$. Hier wird die Konvergenz im Sinne der 
Supremumsnormkonvergenz verstanden und damit handelt es
sich um gleichmäßige Konvergenz, was die Aussage zeigt.
\end{proof}
\printindex
\printbibliography
\end{document}
